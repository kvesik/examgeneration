% Ensure that you compile using XeLaTeX !!! PDFTex has problems with some of the packages used
\documentclass[12pt]{article}
\setlength\parindent{0pt}

\usepackage{parskip}
\usepackage[margin=0.5in]{geometry}
\usepackage{fullpage}
\usepackage{moresize}
\usepackage{graphicx}
\usepackage{caption}
\usepackage{subcaption}
\usepackage{float}
\usepackage{xcolor}
\usepackage{soul}
\usepackage{fontspec}
\setmainfont{Doulos SIL}

\begin{document}

\begin{center}
\textbf{{\color{violet}{\HUGE 20201021 Wednesday\\}}}

\textbf{{\color{violet}{\HUGE ALL EXAMS\\}}}

\end{center}
\newpage

\begin{center}
\textbf{{\color{blue}{\HUGE START OF EXAM\\}}}

\textbf{{\color{blue}{\HUGE Student ID: 62089\\}}}

\textbf{{\color{blue}{\HUGE 4:00\\}}}

\end{center}
\newpage

{\large Question 1}\\

Source: Week 4 Discussion\\

Explain what the given feature’s value is for this class of sounds, and why.\\

{[approximant]}

nasals


\newpage

{\large Question 2}\\

Source: Week 6 Handout, Question 1\\

Would you expect [ʔ] to follow the same pattern of occurring as plain or aspirated as is followed by [p] and [t] and [k]? Explain why or why not.\\

\begin{figure}[H]
\includegraphics{../images/aspiration.png}
\end{figure}

\newpage

\begin{center}
\textbf{{\color{red}{\HUGE END OF EXAM}}}\\

\end{center}
\newpage

\begin{center}
\textbf{{\color{blue}{\HUGE START OF EXAM\\}}}

\textbf{{\color{blue}{\HUGE Student ID: 51967\\}}}

\textbf{{\color{blue}{\HUGE 4:10\\}}}

\end{center}
\newpage

{\large Question 1}\\

Source: Homework 2, Question 3\\

Why should the following two questions have the same answer?\\

\begin{itemize} \item Given the vowel system of Jita, how many bi-syllabic root types would you expect to find for nouns in the language? \item Assuming that the vowel inventory is the same in verbs as it is in nouns, how many bisyllabic root types would you expect to find for verbs in the language? \end{itemize}


\newpage

{\large Question 2}\\

Source: Homework 1, Question 3(a)\\

Could this image be the result of producing the sound represented by the given IPA symbol? Why or why not?\\

{[z]}

\begin{figure}[H]
\includegraphics{../images/staticpalatography_fricative.png}
\end{figure}

\newpage

\begin{center}
\textbf{{\color{red}{\HUGE END OF EXAM}}}\\

\end{center}
\newpage

\begin{center}
\textbf{{\color{blue}{\HUGE START OF EXAM\\}}}

\textbf{{\color{blue}{\HUGE Student ID: 55084\\}}}

\textbf{{\color{blue}{\HUGE 4:20\\}}}

\end{center}
\newpage

{\large Question 1}\\

Source: Homework 1, Question 3(a)\\

Could this image be the result of producing the sound represented by the given IPA symbol? Why or why not?\\

{[t͡ʃ]}

\begin{figure}[H]
\includegraphics{../images/staticpalatography_fricative.png}
\end{figure}

\newpage

{\large Question 2}\\

Source: Week 6 Handout, Question 7\\

Explain how you would determine the phonological relationship between these two sounds (given below) in this dataset.\\

{[d]} and {[n]}

\begin{figure}[H]
\includegraphics{../images/canadianfrench.png}
\end{figure}

\newpage

\begin{center}
\textbf{{\color{red}{\HUGE END OF EXAM}}}\\

\end{center}
\newpage

\begin{center}
\textbf{{\color{blue}{\HUGE START OF EXAM\\}}}

\textbf{{\color{blue}{\HUGE Student ID: 60467\\}}}

\textbf{{\color{blue}{\HUGE 4:30\\}}}

\end{center}
\newpage

{\large Question 1}\\

Source: Week 5 Handout, Question 6\\

If I gave you a new word in Malto, [di\_\_u], would it be possible to predict whether it's [d] or [ɖ] that goes in the blank? Explain why or why not.\\

\begin{figure}[H]
\includegraphics{../images/malto.png}
\end{figure}

\newpage

{\large Question 2}\\

Source: Week 4 Handout, Part II, Question 3\\

Explain how you would figure out what the Luiseño form is for the morpheme whose meaning is given below. (To be clear: you do NOT need to give me the form itself -- just explain the process of figuring it out.)\\

‘third person masc. object’ (‘him’)

\begin{figure}[H]
\includegraphics{../images/luiseno.png}
\end{figure}

\newpage

\begin{center}
\textbf{{\color{red}{\HUGE END OF EXAM}}}\\

\end{center}
\newpage

\begin{center}
\textbf{{\color{blue}{\HUGE START OF EXAM\\}}}

\textbf{{\color{blue}{\HUGE Student ID: 74654\\}}}

\textbf{{\color{blue}{\HUGE 4:40\\}}}

\end{center}
\newpage

{\large Question 1}\\

Source: Week 3 Handout, Question 3\\

Explain why the additional vowel below either does or does not belong in the phonetic natural class defined by the original set of SNAE vowels.\\

Original set: {[æ]}, {[ɑ]}

Addition: {[ɑɪ]}


\newpage

{\large Question 2}\\

Source: Week 4 Handout, Part II, Question 3\\

Explain how you would figure out what the Luiseño form is for the morpheme whose meaning is given below. (To be clear: you do NOT need to give me the form itself -- just explain the process of figuring it out.)\\

‘drink’

\begin{figure}[H]
\includegraphics{../images/luiseno.png}
\end{figure}

\newpage

\begin{center}
\textbf{{\color{red}{\HUGE END OF EXAM}}}\\

\end{center}
\newpage

\begin{center}
\textbf{{\color{blue}{\HUGE START OF EXAM\\}}}

\textbf{{\color{blue}{\HUGE Student ID: 43672\\}}}

\textbf{{\color{blue}{\HUGE 4:50\\}}}

\end{center}
\newpage

{\large Question 1}\\

Source: Homework 1, Question 3(a)\\

Could this image be the result of producing the sound represented by the given IPA symbol? Why or why not?\\

{[d]}

\begin{figure}[H]
\includegraphics{../images/staticpalatography_fricative.png}
\end{figure}

\newpage

{\large Question 2}\\

Source: Week 4 Handout, Part II, Question 4\\

Explain how you could do morphological analysis on a signed language.\\


\newpage

\begin{center}
\textbf{{\color{red}{\HUGE END OF EXAM}}}\\

\end{center}
\newpage

\end{document}

