% Ensure that you compile using XeLaTeX !!! PDFTex has problems with some of the packages used
\documentclass[12pt]{article}
\setlength\parindent{0pt}

\usepackage{parskip}
\usepackage[margin=0.5in]{geometry}
\usepackage{fullpage}
\usepackage{moresize}
\usepackage{graphicx}
\usepackage{caption}
\usepackage{subcaption}
\usepackage{float}
\usepackage{xcolor}
\usepackage{soul}
\usepackage{fontspec}
\setmainfont{Doulos SIL}

\begin{document}

\begin{center}
\textbf{{\color{violet}{\HUGE 20201117 Tuesday\\}}}

\textbf{{\color{violet}{\HUGE ALL EXAMS\\}}}

\end{center}
\newpage

\begin{center}
\textbf{{\color{blue}{\HUGE START OF EXAM\\}}}

\textbf{{\color{blue}{\HUGE Student ID: 42792\\}}}

\textbf{{\color{blue}{\HUGE 4:00\\}}}

\end{center}
\newpage

{\large Question 1}\\

Topic: Other (pre-midterm)\\
Source: Homework 2, Question 2\\

What would this Klingon phrase below be in English? How do you know?\\

{[pɑdɑq]}

\begin{figure}[H]
\includegraphics{../images/klingon.png}
\end{figure}

\newpage

{\large Question 2}\\

Topic: Transcription\\
Source: Week 2 Handout, Part II\\

Is this a reasonable transcription of this word? Explain why.\\

<choose>: {[t͡ʃuz]}


\newpage

\begin{center}
\textbf{{\color{red}{\HUGE END OF EXAM}}}\\

\end{center}
\newpage

\begin{center}
\textbf{{\color{blue}{\HUGE START OF EXAM\\}}}

\textbf{{\color{blue}{\HUGE Student ID: 94549\\}}}

\textbf{{\color{blue}{\HUGE 4:10\\}}}

\end{center}
\newpage

{\large Question 1}\\

Topic: Phonological Features\\
Source: Week 4 Discussion\\

Explain what the given feature’s value is for this class of sounds, and why.\\

{[LABIAL]}

interdentals


\newpage

{\large Question 2}\\

Topic: Articulatory Phonetics\\
Source: Week 3 Handout, Question 3\\

Explain why the additional vowel below either does or does not belong in the phonetic natural class defined by the original set of SNAE vowels.\\

Original set: {[u]}, {[ʊ]}, {[oʊ]}, {[ɔ]}

Addition: {[ɔɪ]}


\newpage

\begin{center}
\textbf{{\color{red}{\HUGE END OF EXAM}}}\\

\end{center}
\newpage

\begin{center}
\textbf{{\color{blue}{\HUGE START OF EXAM\\}}}

\textbf{{\color{blue}{\HUGE Student ID: 59669\\}}}

\textbf{{\color{blue}{\HUGE 4:20\\}}}

\end{center}
\newpage

{\large Question 1}\\

Topic: Articulatory Phonetics\\
Source: Week 3 Discussion\\

Describe what the tongue would do / where it would move during each of the vowels in this word.\\

<waitlist>


\newpage

{\large Question 2}\\

Topic: Other (pre-midterm)\\
Source: Week 5 \& 6 Handouts\\

Explain how you could analyze this dataset in terms of sequential patterns vs. paradigmatic patterns.\\

\begin{figure}[H]
\includegraphics{../images/ukrainian.png}
\end{figure}

\newpage

\begin{center}
\textbf{{\color{red}{\HUGE END OF EXAM}}}\\

\end{center}
\newpage

\begin{center}
\textbf{{\color{blue}{\HUGE START OF EXAM\\}}}

\textbf{{\color{blue}{\HUGE Student ID: 67125\\}}}

\textbf{{\color{blue}{\HUGE 4:30\\}}}

\end{center}
\newpage

{\large Question 1}\\

Topic: Transcription\\
Source: Week 2 Handout, Part II\\

Is this a reasonable transcription of this word? Explain why.\\

<mine>: {[mɑɪn]}


\newpage

{\large Question 2}\\

Topic: Other (pre-midterm)\\
Source: Homework 2, Question 2\\

What would this Klingon phrase below be in English? How do you know?\\

{[pɑdɑq]}

\begin{figure}[H]
\includegraphics{../images/klingon.png}
\end{figure}

\newpage

\begin{center}
\textbf{{\color{red}{\HUGE END OF EXAM}}}\\

\end{center}
\newpage

\begin{center}
\textbf{{\color{blue}{\HUGE START OF EXAM\\}}}

\textbf{{\color{blue}{\HUGE Student ID: 96220\\}}}

\textbf{{\color{blue}{\HUGE 4:40\\}}}

\end{center}
\newpage

{\large Question 1}\\

Topic: Articulatory Phonetics\\
Source: Week 3 Handout, Question 3\\

Explain why the additional vowel below either does or does not belong in the phonetic natural class defined by the original set of SNAE vowels.\\

Original set: {[u]}, {[ʊ]}, {[oʊ]}, {[ɔ]}

Addition: {[ɔɪ]}


\newpage

{\large Question 2}\\

Topic: Phonological Features\\
Source: Homework 2, Question 1\\

Explain which sound should be removed to make this a natural class (assuming SNAE, except that there are no diphthongs, no [ə] or [ʌ], no syllabic consonants, and no [w̥]), and what the minimum set of features would be to describe the resulting natural class.\\

{[i]}, {[ɪ]}, {[e]}, {[ɛ]}, {[æ]}, {[ɑ]}, {[ɔ]}, {[o]}, {[ʊ]}, {[u]}, {[ʒ]}, {[k]}, {[ɡ]}, {[ŋ]}, {[w]}


\newpage

\begin{center}
\textbf{{\color{red}{\HUGE END OF EXAM}}}\\

\end{center}
\newpage

\begin{center}
\textbf{{\color{blue}{\HUGE START OF EXAM\\}}}

\textbf{{\color{blue}{\HUGE Student ID: 16922\\}}}

\textbf{{\color{blue}{\HUGE 4:50\\}}}

\end{center}
\newpage

{\large Question 1}\\

Topic: Other (pre-midterm)\\
Source: Week 4 Handout, Part II, Question 3\\

Explain how you would figure out what the Luiseño form is for the morpheme whose meaning is given below. (To be clear: you do NOT need to give me the form itself -- just explain the process of figuring it out.)\\

‘first person plural object’ (‘us’)

\begin{figure}[H]
\includegraphics{../images/luiseno.png}
\end{figure}

\newpage

{\large Question 2}\\

Topic: Phonological Features\\
Source: Week 4 Discussion\\

Explain why phonological features are used instead of phonetic characteristics in analyzing datasets.\\


\newpage

\begin{center}
\textbf{{\color{red}{\HUGE END OF EXAM}}}\\

\end{center}
\newpage

\end{document}

