% Ensure that you compile using XeLaTeX !!! PDFTex has problems with some of the packages used
\documentclass[12pt]{article}
\setlength\parindent{0pt}

\usepackage{parskip}
\usepackage[margin=0.5in]{geometry}
\usepackage{fullpage}
\usepackage{moresize}
\usepackage{graphicx}
\usepackage{caption}
\usepackage{subcaption}
\usepackage{float}
\usepackage{xcolor}
\usepackage{soul}
\usepackage{fontspec}
\setmainfont{Doulos SIL}

\begin{document}

\begin{center}
\textbf{{\color{violet}{\HUGE 20201103 Tuesday\\}}}

\textbf{{\color{violet}{\HUGE ALL EXAMS (with notes)\\}}}

\end{center}
\newpage

\begin{center}
\textbf{{\color{blue}{\HUGE START OF EXAM\\}}}

\textbf{{\color{blue}{\HUGE Student ID: 43803\\}}}

\textbf{{\color{blue}{\HUGE 4:00\\}}}

\end{center}
\newpage

{\large Question 1}\\

Topic: Articulatory Phonetics\\
Source: Homework 1, Question 3(a)\\

Could this image be the result of producing the sound represented by the given IPA symbol? Why or why not?\\

{[t͡ʃ]}

\begin{figure}[H]
\includegraphics{../images/staticpalatography_fricative.png}
\end{figure}

~\\
INSTRUCTOR NOTES: no


\vfill
Excellent (3) ~~~ Good (2.2) ~~~ Fair (1.7) ~~~ Poor (0)
\newpage

{\large Question 2}\\

Topic: Skewed Distributions\\
Source: Week 5 Handout, Question 3\\

What evidence is there that there is a pattern in these data, assuming that these are the only CV and VC sequences that occur in some language?\\

{[sa]}, {[ʃi]}, {[za]}, {[ʒi]}, {[as]}, {[iʃ]}, {[az]}, {[iʒ]}


~\\
INSTRUCTOR NOTES: (the palatal sounds occur with the high vowel, while the alveolar sounds occur with the low vowel)


\vfill
Excellent (3) ~~~ Good (2.2) ~~~ Fair (1.7) ~~~ Poor (0)
\newpage

\begin{center}
\textbf{{\color{red}{\HUGE END OF EXAM}}}\\

\end{center}
\newpage

\begin{center}
\textbf{{\color{blue}{\HUGE START OF EXAM\\}}}

\textbf{{\color{blue}{\HUGE Student ID: 18870\\}}}

\textbf{{\color{blue}{\HUGE 4:10\\}}}

\end{center}
\newpage

{\large Question 1}\\

Topic: Articulatory Phonetics\\
Source: Quiz 2, Question 6\\

In the pronunciation of this word, which sounds are obstruents and which are sonorants? Explain your answer.\\

<language>


~\\
INSTRUCTOR NOTES: [læŋɡwɪdʒ] -- sonorants: [læŋwɪ] and obstruents: [ɡdʒ]


\vfill
Excellent (3) ~~~ Good (2.2) ~~~ Fair (1.7) ~~~ Poor (0)
\newpage

{\large Question 2}\\

Topic: Transcription\\
Source: Week 2 Handout, Part II\\

Is this a reasonable transcription of this word? Explain why.\\

<choose>: {[t͡ʃuz]}


~\\
INSTRUCTOR NOTES: yes


\vfill
Excellent (3) ~~~ Good (2.2) ~~~ Fair (1.7) ~~~ Poor (0)
\newpage

\begin{center}
\textbf{{\color{red}{\HUGE END OF EXAM}}}\\

\end{center}
\newpage

\begin{center}
\textbf{{\color{blue}{\HUGE START OF EXAM\\}}}

\textbf{{\color{blue}{\HUGE Student ID: empty\\}}}

\textbf{{\color{blue}{\HUGE 4:20\\}}}

\end{center}
\newpage

\begin{center}
\textbf{{\color{blue}{\HUGE START OF EXAM\\}}}

\textbf{{\color{blue}{\HUGE Student ID: 28926\\}}}

\textbf{{\color{blue}{\HUGE 4:30\\}}}

\end{center}
\newpage

{\large Question 1}\\

Topic: Phonological Features\\
Source: Quiz 3, Question 12\\

Explain how you figure out which feature is involved in the process of umlaut shown below.\\

\begin{figure}[H]
\includegraphics{../images/dutch.png}
\end{figure}

~\\
INSTRUCTOR NOTES: we look to see which vowels are affected, and compare them to see which feature is DIFFERENT (not e.g. what features they share); so since the vowels in the singular and plural are identical except that the singular forms are back and the plural are front, it's the feature [back] that is relevant / changing / involved (not e.g. the feature [round] just because all of the vowels are round)


\vfill
Excellent (3) ~~~ Good (2.2) ~~~ Fair (1.7) ~~~ Poor (0)
\newpage

{\large Question 2}\\

Topic: Articulatory Phonetics\\
Source: Homework 1, Question 3(b)\\

Explain why this is or is not a complete phonetic natural class in standard North American English.\\

{[ɔ]}, {[ʊ]}, {[u]}, {[oʊ]}


~\\
INSTRUCTOR NOTES: yes (all back rounded vowels)


\vfill
Excellent (3) ~~~ Good (2.2) ~~~ Fair (1.7) ~~~ Poor (0)
\newpage

\begin{center}
\textbf{{\color{red}{\HUGE END OF EXAM}}}\\

\end{center}
\newpage

\begin{center}
\textbf{{\color{blue}{\HUGE START OF EXAM\\}}}

\textbf{{\color{blue}{\HUGE Student ID: 83324\\}}}

\textbf{{\color{blue}{\HUGE 4:40\\}}}

\end{center}
\newpage

{\large Question 1}\\

Topic: Transcription\\
Source: Week 2 Handout, Part II\\

Is this a reasonable transcription of this word? Explain why.\\

<choose>: {[t͡ʃuz]}


~\\
INSTRUCTOR NOTES: yes


\vfill
Excellent (3) ~~~ Good (2.2) ~~~ Fair (1.7) ~~~ Poor (0)
\newpage

{\large Question 2}\\

Topic: Articulatory Phonetics\\
Source: Quiz 2, Question 7\\

Why might more than one of the descriptions given truthfully apply to the sound represented by the underlined letter, and why is one of them actually better than the other?\\

<a\underline{w}ay>

\begin{itemize} \item prevocalic obstruent \item prevocalic sonorant \item postvocalic obstruent \item postvocalic sonorant \item intervocalic obstruent \item intervocalic sonorant \end{itemize}


~\\
INSTRUCTOR NOTES: prevocalic and *intervocalic* sonorant


\vfill
Excellent (3) ~~~ Good (2.2) ~~~ Fair (1.7) ~~~ Poor (0)
\newpage

\begin{center}
\textbf{{\color{red}{\HUGE END OF EXAM}}}\\

\end{center}
\newpage

\begin{center}
\textbf{{\color{blue}{\HUGE START OF EXAM\\}}}

\textbf{{\color{blue}{\HUGE Student ID: 80815\\}}}

\textbf{{\color{blue}{\HUGE 4:50\\}}}

\end{center}
\newpage

{\large Question 1}\\

Topic: Other (pre-midterm)\\
Source: Week 5 \& 6 Handouts\\

Explain how you could analyze this dataset in terms of sequential patterns vs. paradigmatic patterns.\\

\begin{figure}[H]
\includegraphics{../images/ukrainian.png}
\end{figure}

~\\
INSTRUCTOR NOTES: 


\vfill
Excellent (3) ~~~ Good (2.2) ~~~ Fair (1.7) ~~~ Poor (0)
\newpage

{\large Question 2}\\

Topic: Skewed Distributions\\
Source: Week 5 Handout, Question 6\\

What would be a good description of the pattern in Malto? What characteristics make that a good description?\\

\begin{figure}[H]
\includegraphics{../images/malto.png}
\end{figure}

~\\
INSTRUCTOR NOTES: It is not possible to have stops of different places of articulation co-occurring in a single word. Instead, if there are two stops, both must be dental, or both must be retroflex. For example, it’s possible to have a word with two dental stops, as in [tot] ‘to hurry’ (even if they disagree in voicing, as in [tind] ‘to feed’), or a word with two retroflex stops, as in [ɖebɖe] ‘crooked’ (again, regardless of voicing, as in [ɖeʈa] ‘corn cob’). But there are no words that have one dental and one retroflex stop, in either order, regardless of voicing. (accurate, generalizations, concrete examples)


\vfill
Excellent (3) ~~~ Good (2.2) ~~~ Fair (1.7) ~~~ Poor (0)
\newpage

\begin{center}
\textbf{{\color{red}{\HUGE END OF EXAM}}}\\

\end{center}
\newpage

\end{document}

