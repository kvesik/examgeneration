% Ensure that you compile using XeLaTeX !!! PDFTex has problems with some of the packages used
\documentclass[12pt]{article}
\setlength\parindent{0pt}

\usepackage{parskip}
\usepackage[margin=0.5in]{geometry}
\usepackage{fullpage}
\usepackage{moresize}
\usepackage{graphicx}
\usepackage{caption}
\usepackage{subcaption}
\usepackage{float}
\usepackage{xcolor}
\usepackage{soul}
\usepackage{fontspec}
\setmainfont{Doulos SIL}

\begin{document}

\begin{center}
\textbf{{\color{violet}{\HUGE 20201104 Wednesday\\}}}

\textbf{{\color{violet}{\HUGE ALL EXAMS\\}}}

\end{center}
\newpage

\begin{center}
\textbf{{\color{blue}{\HUGE START OF EXAM\\}}}

\textbf{{\color{blue}{\HUGE Student ID: 44746\\}}}

\textbf{{\color{blue}{\HUGE 4:00\\}}}

\end{center}
\newpage

{\large Question 1}\\

Topic: Other (pre-midterm)\\
Source: Homework 2, Question 2\\

What would this Klingon phrase below be in English? How do you know?\\

{[pɑdɑq]}

\begin{figure}[H]
\includegraphics{../images/klingon.png}
\end{figure}

\newpage

{\large Question 2}\\

Topic: Transcription\\
Source: Week 2 Handout, Part II\\

Is this a reasonable transcription of this word? Explain why.\\

<philosophy>: {[fəlɑsəfi]}


\newpage

\begin{center}
\textbf{{\color{red}{\HUGE END OF EXAM}}}\\

\end{center}
\newpage

\begin{center}
\textbf{{\color{blue}{\HUGE START OF EXAM\\}}}

\textbf{{\color{blue}{\HUGE Student ID: 74752\\}}}

\textbf{{\color{blue}{\HUGE 4:10\\}}}

\end{center}
\newpage

{\large Question 1}\\

Topic: Transcription\\
Source: Week 2 Handout, Part II\\

Is this a reasonable transcription of this word? Explain why.\\

<wimp>: {[wimp]}


\newpage

{\large Question 2}\\

Topic: Articulatory Phonetics\\
Source: Quiz 2, Question 6\\

In the pronunciation of this word, which sounds are obstruents and which are sonorants? Explain your answer.\\

<sonorant>


\newpage

\begin{center}
\textbf{{\color{red}{\HUGE END OF EXAM}}}\\

\end{center}
\newpage

\begin{center}
\textbf{{\color{blue}{\HUGE START OF EXAM\\}}}

\textbf{{\color{blue}{\HUGE Student ID: 94263\\}}}

\textbf{{\color{blue}{\HUGE 4:20\\}}}

\end{center}
\newpage

{\large Question 1}\\

Topic: Skewed Distributions\\
Source: Week 5 Handout, Question 5\\

Explain why looking for patterns with consonants and vowels is a more reasonable approach to pattern finding in this dataset than looking for patterns with respect to all of the individual sounds in Ukrainian.\\

\begin{figure}[H]
\includegraphics{../images/ukrainian.png}
\end{figure}

\newpage

{\large Question 2}\\

Topic: Transcription\\
Source: Week 2 Handout, Part II, Question 11\\

How would this word be transcribed?\\ (Kathleen will then ask a follow-up question about your transcription.)\\

<goat>


\newpage

\begin{center}
\textbf{{\color{red}{\HUGE END OF EXAM}}}\\

\end{center}
\newpage

\begin{center}
\textbf{{\color{blue}{\HUGE START OF EXAM\\}}}

\textbf{{\color{blue}{\HUGE Student ID: 70875\\}}}

\textbf{{\color{blue}{\HUGE 4:30\\}}}

\end{center}
\newpage

{\large Question 1}\\

Topic: Articulatory Phonetics\\
Source: Week 3 Discussion\\

Assuming a Standard North American English inventory, does this vowel need to have tenseness specified if you're giving a prose description? Why or why not?\\

{[i]}


\newpage

{\large Question 2}\\

Topic: Other (pre-midterm)\\
Source: Week 5 \& 6 Handouts\\

Explain how you could analyze this dataset in terms of sequential patterns vs. paradigmatic patterns.\\

\begin{figure}[H]
\includegraphics{../images/ukrainian.png}
\end{figure}

\newpage

\begin{center}
\textbf{{\color{red}{\HUGE END OF EXAM}}}\\

\end{center}
\newpage

\begin{center}
\textbf{{\color{blue}{\HUGE START OF EXAM\\}}}

\textbf{{\color{blue}{\HUGE Student ID: 12377\\}}}

\textbf{{\color{blue}{\HUGE 4:40\\}}}

\end{center}
\newpage

{\large Question 1}\\

Topic: Articulatory Phonetics\\
Source: Quiz 2, Question 7\\

Why might more than one of the descriptions given truthfully apply to the sound represented by the underlined letter, and why is one of them actually better than the other?\\

<a\underline{w}ay>

\begin{itemize} \item prevocalic obstruent \item prevocalic sonorant \item postvocalic obstruent \item postvocalic sonorant \item intervocalic obstruent \item intervocalic sonorant \end{itemize}


\newpage

{\large Question 2}\\

Topic: Skewed Distributions\\
Source: Week 5 Handout, Question 5\\

Explain why looking for patterns with consonants and vowels is a more reasonable approach to pattern finding in this dataset than looking for patterns with respect to all of the individual sounds in Ukrainian.\\

\begin{figure}[H]
\includegraphics{../images/ukrainian.png}
\end{figure}

\newpage

\begin{center}
\textbf{{\color{red}{\HUGE END OF EXAM}}}\\

\end{center}
\newpage

\begin{center}
\textbf{{\color{blue}{\HUGE START OF EXAM\\}}}

\textbf{{\color{blue}{\HUGE Student ID: 36593\\}}}

\textbf{{\color{blue}{\HUGE 4:50\\}}}

\end{center}
\newpage

{\large Question 1}\\

Topic: Skewed Distributions\\
Source: Week 5 Handout, Question 5\\

Explain why looking for patterns with consonants and vowels is a more reasonable approach to pattern finding in this dataset than looking for patterns with respect to all of the individual sounds in Ukrainian.\\

\begin{figure}[H]
\includegraphics{../images/ukrainian.png}
\end{figure}

\newpage

{\large Question 2}\\

Topic: Other (pre-midterm)\\
Source: Week 2 Handout, Part I, Question 12\\

Is this question about phonetics or phonology, and why? (To be clear: you do NOT need to answer the question itself -- just tell me whether it's a question about phonetics or phonology.)\\

What’s the difference in sound between the words <but> and <cut> in English? Is that the same difference as between <bit> and <kit>? Why or why not?


\newpage

\begin{center}
\textbf{{\color{red}{\HUGE END OF EXAM}}}\\

\end{center}
\newpage

\end{document}

