% Ensure that you compile using XeLaTeX !!! PDFTex has problems with some of the packages used
\documentclass[12pt]{article}
\setlength\parindent{0pt}

\usepackage{parskip}
\usepackage[margin=0.5in]{geometry}
\usepackage{fullpage}
\usepackage{moresize}
\usepackage{graphicx}
\usepackage{caption}
\usepackage{subcaption}
\usepackage{float}
\usepackage{xcolor}
\usepackage{soul}
\usepackage{fontspec}
\setmainfont{Doulos SIL}

\begin{document}

\begin{center}
\textbf{{\color{violet}{\HUGE 20201006 Tuesday\\}}}

\textbf{{\color{violet}{\HUGE ALL EXAMS (with notes)\\}}}

\end{center}
\newpage

\begin{center}
\textbf{{\color{blue}{\HUGE START OF EXAM\\}}}

\textbf{{\color{blue}{\HUGE Student ID: 15082\\}}}

\textbf{{\color{blue}{\HUGE 9:00\\}}}

\end{center}
\newpage

{\large Question 1}\\

Topic: Other (pre-midterm)\\
Source: Week 4 Handout, Part II, Question 3\\

Explain how you would figure out what the Luiseño form is for the morpheme whose meaning is given below.\\

‘drink’

\begin{figure}[H]
\includegraphics{../images/luiseno.png}
\end{figure}

~\\
INSTRUCTOR NOTES: ([páaʔi])


\vfill
Excellent (3) ~~~ Good (2.2) ~~~ Fair (1.7) ~~~ Poor (0)
\newpage

{\large Question 2}\\

Topic: Transcription\\
Source: Week 2 Handout, Part II\\

Is this a reasonable transcription of this word? Explain why.\\

<shows>: {[ʃoʊs]}


~\\
INSTRUCTOR NOTES: no, [z]


\vfill
Excellent (3) ~~~ Good (2.2) ~~~ Fair (1.7) ~~~ Poor (0)
\newpage

\begin{center}
\textbf{{\color{red}{\HUGE END OF EXAM}}}\\

\end{center}
\newpage

\begin{center}
\textbf{{\color{blue}{\HUGE START OF EXAM\\}}}

\textbf{{\color{blue}{\HUGE Student ID: 47906\\}}}

\textbf{{\color{blue}{\HUGE 9:10\\}}}

\end{center}
\newpage

{\large Question 1}\\

Topic: Other (pre-midterm)\\
Source: Week 4 Handout, Part II, Question 3\\

Explain how you would figure out what the Luiseño form is for the morpheme whose meaning is given below.\\

‘sun’ (or ‘the sun’)

\begin{figure}[H]
\includegraphics{../images/luiseno.png}
\end{figure}

~\\
INSTRUCTOR NOTES: ([temét])


\vfill
Excellent (3) ~~~ Good (2.2) ~~~ Fair (1.7) ~~~ Poor (0)
\newpage

{\large Question 2}\\

Topic: Phonological Features\\
Source: Quiz 3, Question 12\\

Explain how you figure out which feature is involved in the process of umlaut.\\

\begin{figure}[H]
\includegraphics{../images/dutch.png}
\end{figure}

~\\
INSTRUCTOR NOTES: we look to see which vowels are affected, and compare them to see which feature is DIFFERENT (not e.g. what features they share); so since the vowels in the singular and plural are identical except that the singular forms are back and the plural are front, it's the feature [back] that is relevant / changing / involved (not e.g. the feature [round] just because all of the vowels are round)


\vfill
Excellent (3) ~~~ Good (2.2) ~~~ Fair (1.7) ~~~ Poor (0)
\newpage

\begin{center}
\textbf{{\color{red}{\HUGE END OF EXAM}}}\\

\end{center}
\newpage

\begin{center}
\textbf{{\color{blue}{\HUGE START OF EXAM\\}}}

\textbf{{\color{blue}{\HUGE Student ID: empty\\}}}

\textbf{{\color{blue}{\HUGE 9:20\\}}}

\end{center}
\newpage

\begin{center}
\textbf{{\color{blue}{\HUGE START OF EXAM\\}}}

\textbf{{\color{blue}{\HUGE Student ID: 54112\\}}}

\textbf{{\color{blue}{\HUGE 9:30\\}}}

\end{center}
\newpage

{\large Question 1}\\

Topic: Other (pre-midterm)\\
Source: Week 4 Handout, Part II, Question 2(iv)\\

Explain how you would figure out the Swahili word for this English gloss.\\

‘I wanted them.’

\begin{figure}[H]
\includegraphics{../images/swahiliverbs.png}
\end{figure}

~\\
INSTRUCTOR NOTES: ([niliwataka])


\vfill
Excellent (3) ~~~ Good (2.2) ~~~ Fair (1.7) ~~~ Poor (0)
\newpage

{\large Question 2}\\

Topic: Transcription\\
Source: Week 2 Handout, Part II\\

Is this a reasonable transcription of this word? Explain why.\\

<philosophy>: {[fəlɑsəfi]}


~\\
INSTRUCTOR NOTES: yes


\vfill
Excellent (3) ~~~ Good (2.2) ~~~ Fair (1.7) ~~~ Poor (0)
\newpage

\begin{center}
\textbf{{\color{red}{\HUGE END OF EXAM}}}\\

\end{center}
\newpage

\begin{center}
\textbf{{\color{blue}{\HUGE START OF EXAM\\}}}

\textbf{{\color{blue}{\HUGE Student ID: 89930\\}}}

\textbf{{\color{blue}{\HUGE 9:40\\}}}

\end{center}
\newpage

{\large Question 1}\\

Topic: Other (pre-midterm)\\
Source: Week 2 Handout, Part I, Question 8\\

Is this question about phonetics or phonology, and why? (To be clear: you do NOT need to answer the question itself -- just tell me whether it's a question about phonetics or phonology.)\\

Consider the following two words from American Sign Language. The first one means LUCKY, while the second means SMART. How would you describe the difference between the ``pronunciation'' (articulation) of these two words? Note that in each case, the image to the left is the starting position of the sign, while the one to the right is the ending position.

\begin{figure}[H]
\includegraphics{../images/asl_lucky.png}
\end{figure}
\begin{figure}[H]
\includegraphics{../images/asl_smart.png}
\end{figure}

~\\
INSTRUCTOR NOTES: phonetics


\vfill
Excellent (3) ~~~ Good (2.2) ~~~ Fair (1.7) ~~~ Poor (0)
\newpage

{\large Question 2}\\

Topic: Articulatory Phonetics\\
Source: Week 3 Handout, Question 3\\

Explain why the additional vowel below either does or does not belong in the phonetic natural class defined by the original set of SNAE vowels.\\

Original set: {[æ]}, {[ɑ]}

Addition: {[ɑʊ]}


~\\
INSTRUCTOR NOTES: should recognize that there's more than one vowel sound, which makes it somewhat difficult to categorize; best answers will say that the diphthong is crucially a diphthong and so can't also go in this class


\vfill
Excellent (3) ~~~ Good (2.2) ~~~ Fair (1.7) ~~~ Poor (0)
\newpage

\begin{center}
\textbf{{\color{red}{\HUGE END OF EXAM}}}\\

\end{center}
\newpage

\begin{center}
\textbf{{\color{blue}{\HUGE START OF EXAM\\}}}

\textbf{{\color{blue}{\HUGE Student ID: 97077\\}}}

\textbf{{\color{blue}{\HUGE 9:50\\}}}

\end{center}
\newpage

{\large Question 1}\\

Topic: Other (pre-midterm)\\
Source: Week 4 Handout, Part II, Question 2(iv)\\

Explain how you would figure out the Swahili word for this English gloss.\\

‘You (sg.) are annoying me.’

\begin{figure}[H]
\includegraphics{../images/swahiliverbs.png}
\end{figure}

~\\
INSTRUCTOR NOTES: ([unanisumbua])


\vfill
Excellent (3) ~~~ Good (2.2) ~~~ Fair (1.7) ~~~ Poor (0)
\newpage

{\large Question 2}\\

Topic: Phonological Features\\
Source: Quiz 3, Question 12\\

Explain how you figure out which feature is involved in the process of umlaut.\\

\begin{figure}[H]
\includegraphics{../images/dutch.png}
\end{figure}

~\\
INSTRUCTOR NOTES: we look to see which vowels are affected, and compare them to see which feature is DIFFERENT (not e.g. what features they share); so since the vowels in the singular and plural are identical except that the singular forms are back and the plural are front, it's the feature [back] that is relevant / changing / involved (not e.g. the feature [round] just because all of the vowels are round)


\vfill
Excellent (3) ~~~ Good (2.2) ~~~ Fair (1.7) ~~~ Poor (0)
\newpage

\begin{center}
\textbf{{\color{red}{\HUGE END OF EXAM}}}\\

\end{center}
\newpage

\end{document}

