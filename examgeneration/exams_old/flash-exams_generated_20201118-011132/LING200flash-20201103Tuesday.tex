% Ensure that you compile using XeLaTeX !!! PDFTex has problems with some of the packages used
\documentclass[12pt]{article}
\setlength\parindent{0pt}

\usepackage{parskip}
\usepackage[margin=0.5in]{geometry}
\usepackage{fullpage}
\usepackage{moresize}
\usepackage{graphicx}
\usepackage{caption}
\usepackage{subcaption}
\usepackage{float}
\usepackage{xcolor}
\usepackage{soul}
\usepackage{fontspec}
\setmainfont{Doulos SIL}

\begin{document}

\begin{center}
\textbf{{\color{violet}{\HUGE 20201103 Tuesday\\}}}

\textbf{{\color{violet}{\HUGE ALL EXAMS\\}}}

\end{center}
\newpage

\begin{center}
\textbf{{\color{blue}{\HUGE START OF EXAM\\}}}

\textbf{{\color{blue}{\HUGE Student ID: 43803\\}}}

\textbf{{\color{blue}{\HUGE 4:00\\}}}

\end{center}
\newpage

{\large Question 1}\\

Topic: Transcription\\
Source: Week 2 Handout, Part II, Question 11\\

How would this word be transcribed?\\ (Kathleen will then ask a follow-up question about your transcription.)\\

<juice>


\newpage

{\large Question 2}\\

Topic: Other (pre-midterm)\\
Source: Week 5 \& 6 Handouts\\

Explain how you could analyze this dataset in terms of sequential patterns vs. paradigmatic patterns.\\

\begin{figure}[H]
\includegraphics{../images/ukrainian.png}
\end{figure}

\newpage

\begin{center}
\textbf{{\color{red}{\HUGE END OF EXAM}}}\\

\end{center}
\newpage

\begin{center}
\textbf{{\color{blue}{\HUGE START OF EXAM\\}}}

\textbf{{\color{blue}{\HUGE Student ID: 18870\\}}}

\textbf{{\color{blue}{\HUGE 4:10\\}}}

\end{center}
\newpage

{\large Question 1}\\

Topic: Phonological Features\\
Source: Homework 2, Question 1\\

Explain which sound should be removed to make this a natural class (assuming SNAE, except that there are no diphthongs, no [ə] or [ʌ], no syllabic consonants, and no [w̥]), and what the minimum set of features would be to describe the resulting natural class.\\

{[i]}, {[ɪ]}, {[ɛ]}, {[u]}, {[ʊ]}


\newpage

{\large Question 2}\\

Topic: Other (pre-midterm)\\
Source: Week 5 \& 6 Handouts\\

Explain how you could analyze this dataset in terms of sequential patterns vs. paradigmatic patterns.\\

\begin{figure}[H]
\includegraphics{../images/ukrainian.png}
\end{figure}

\newpage

\begin{center}
\textbf{{\color{red}{\HUGE END OF EXAM}}}\\

\end{center}
\newpage

\begin{center}
\textbf{{\color{blue}{\HUGE START OF EXAM\\}}}

\textbf{{\color{blue}{\HUGE Student ID: empty\\}}}

\textbf{{\color{blue}{\HUGE 4:20\\}}}

\end{center}
\newpage

\begin{center}
\textbf{{\color{blue}{\HUGE START OF EXAM\\}}}

\textbf{{\color{blue}{\HUGE Student ID: 28926\\}}}

\textbf{{\color{blue}{\HUGE 4:30\\}}}

\end{center}
\newpage

{\large Question 1}\\

Topic: Articulatory Phonetics\\
Source: Homework 1, Question 3(a)\\

Could this image be the result of producing the sound represented by the given IPA symbol? Why or why not?\\

{[t͡ʃ]}

\begin{figure}[H]
\includegraphics{../images/staticpalatography_fricative.png}
\end{figure}

\newpage

{\large Question 2}\\

Topic: Other (pre-midterm)\\
Source: Week 4 Handout, Part II, Question 2(iv)\\

Explain how you would figure out the Swahili word for this English gloss. (To be clear: you do NOT need to give me the Swahili form itself -- just explain the process of figuring it out.)\\

‘You (sg.) are annoying me.’

\begin{figure}[H]
\includegraphics{../images/swahiliverbs.png}
\end{figure}

\newpage

\begin{center}
\textbf{{\color{red}{\HUGE END OF EXAM}}}\\

\end{center}
\newpage

\begin{center}
\textbf{{\color{blue}{\HUGE START OF EXAM\\}}}

\textbf{{\color{blue}{\HUGE Student ID: 83324\\}}}

\textbf{{\color{blue}{\HUGE 4:40\\}}}

\end{center}
\newpage

{\large Question 1}\\

Topic: Other (pre-midterm)\\
Source: Week 4 Handout, Part II, Question 2(iv)\\

Explain how you would figure out the Swahili word for this English gloss. (To be clear: you do NOT need to give me the Swahili form itself -- just explain the process of figuring it out.)\\

‘I wanted them.’

\begin{figure}[H]
\includegraphics{../images/swahiliverbs.png}
\end{figure}

\newpage

{\large Question 2}\\

Topic: Skewed Distributions\\
Source: Week 5 Handout, Question 7\\

Explain how you would go about looking for co-occurrence restrictions in bi-syllabic signs in ASL. (Refer to the data that follows.)\\

\begin{figure}[H]
\includegraphics{../images/ASL_movement.png}
\end{figure}

\newpage

\begin{center}
\textbf{{\color{red}{\HUGE END OF EXAM}}}\\

\end{center}
\newpage

\begin{center}
\textbf{{\color{blue}{\HUGE START OF EXAM\\}}}

\textbf{{\color{blue}{\HUGE Student ID: 80815\\}}}

\textbf{{\color{blue}{\HUGE 4:50\\}}}

\end{center}
\newpage

{\large Question 1}\\

Topic: Phonological Features\\
Source: Homework 2, Question 1\\

Explain which sound should be removed to make this a natural class (assuming SNAE, except that there are no diphthongs, no [ə] or [ʌ], no syllabic consonants, and no [w̥]), and what the minimum set of features would be to describe the resulting natural class.\\

{[v]}, {[z]}, {[ʃ]}, {[ʒ]}, {[ð]}


\newpage

{\large Question 2}\\

Topic: Other (pre-midterm)\\
Source: Week 5 \& 6 Handouts\\

Explain how you could analyze this dataset in terms of sequential patterns vs. paradigmatic patterns.\\

\begin{figure}[H]
\includegraphics{../images/ukrainian.png}
\end{figure}

\newpage

\begin{center}
\textbf{{\color{red}{\HUGE END OF EXAM}}}\\

\end{center}
\newpage

\end{document}

