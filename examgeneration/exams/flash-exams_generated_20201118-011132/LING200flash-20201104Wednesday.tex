% Ensure that you compile using XeLaTeX !!! PDFTex has problems with some of the packages used
\documentclass[12pt]{article}
\setlength\parindent{0pt}

\usepackage{parskip}
\usepackage[margin=0.5in]{geometry}
\usepackage{fullpage}
\usepackage{moresize}
\usepackage{graphicx}
\usepackage{caption}
\usepackage{subcaption}
\usepackage{float}
\usepackage{xcolor}
\usepackage{soul}
\usepackage{fontspec}
\setmainfont{Doulos SIL}

\begin{document}

\begin{center}
\textbf{{\color{violet}{\HUGE 20201104 Wednesday\\}}}

\textbf{{\color{violet}{\HUGE ALL EXAMS\\}}}

\end{center}
\newpage

\begin{center}
\textbf{{\color{blue}{\HUGE START OF EXAM\\}}}

\textbf{{\color{blue}{\HUGE Student ID: 44746\\}}}

\textbf{{\color{blue}{\HUGE 4:00\\}}}

\end{center}
\newpage

{\large Question 1}\\

Topic: Articulatory Phonetics\\
Source: Week 3 Handout, Question 3\\

Explain why the additional vowel below either does or does not belong in the phonetic natural class defined by the original set of SNAE vowels.\\

Original set: {[u]}, {[ʊ]}, {[oʊ]}, {[ɔ]}

Addition: {[ɑʊ]}


\newpage

{\large Question 2}\\

Topic: Phonological Features\\
Source: Week 4 Discussion\\

Explain what the given feature’s value is for this class of sounds, and why.\\

{[approximant]}

nasals


\newpage

\begin{center}
\textbf{{\color{red}{\HUGE END OF EXAM}}}\\

\end{center}
\newpage

\begin{center}
\textbf{{\color{blue}{\HUGE START OF EXAM\\}}}

\textbf{{\color{blue}{\HUGE Student ID: 74752\\}}}

\textbf{{\color{blue}{\HUGE 4:10\\}}}

\end{center}
\newpage

{\large Question 1}\\

Topic: Skewed Distributions\\
Source: Week 5 Handout, Question 6\\

If I gave you a new word in Malto, [di\_\_u], would it be possible to predict whether it's [d] or [ɖ] that goes in the blank? Explain why or why not.\\

\begin{figure}[H]
\includegraphics{../images/malto.png}
\end{figure}

\newpage

{\large Question 2}\\

Topic: Phonological Features\\
Source: Week 4 Discussion\\

Explain what the given feature’s value is for this class of sounds, and why.\\

{[consonantal]}

glides


\newpage

\begin{center}
\textbf{{\color{red}{\HUGE END OF EXAM}}}\\

\end{center}
\newpage

\begin{center}
\textbf{{\color{blue}{\HUGE START OF EXAM\\}}}

\textbf{{\color{blue}{\HUGE Student ID: 94263\\}}}

\textbf{{\color{blue}{\HUGE 4:20\\}}}

\end{center}
\newpage

{\large Question 1}\\

Topic: Other (pre-midterm)\\
Source: Week 4 Handout, Part II, Question 3\\

Explain how you would figure out what the Luiseño form is for the morpheme whose meaning is given below. (To be clear: you do NOT need to give me the form itself -- just explain the process of figuring it out.)\\

‘Future’

\begin{figure}[H]
\includegraphics{../images/luiseno.png}
\end{figure}

\newpage

{\large Question 2}\\

Topic: Skewed Distributions\\
Source: Week 5 Handout, Question 7\\

Explain how you would go about looking for co-occurrence restrictions in bi-syllabic signs in ASL. (Refer to the data that follows.)\\

\begin{figure}[H]
\includegraphics{../images/ASL_movement.png}
\end{figure}

\newpage

\begin{center}
\textbf{{\color{red}{\HUGE END OF EXAM}}}\\

\end{center}
\newpage

\begin{center}
\textbf{{\color{blue}{\HUGE START OF EXAM\\}}}

\textbf{{\color{blue}{\HUGE Student ID: 70875\\}}}

\textbf{{\color{blue}{\HUGE 4:30\\}}}

\end{center}
\newpage

{\large Question 1}\\

Topic: Other (pre-midterm)\\
Source: Week 4 Handout, Part II, Question 3\\

Explain how you would figure out what the Luiseño form is for the morpheme whose meaning is given below. (To be clear: you do NOT need to give me the form itself -- just explain the process of figuring it out.)\\

‘walk’

\begin{figure}[H]
\includegraphics{../images/luiseno.png}
\end{figure}

\newpage

{\large Question 2}\\

Topic: Articulatory Phonetics\\
Source: Homework 1, Question 3(a)\\

Could this image be the result of producing the sound represented by the given IPA symbol? Why or why not?\\

{[t͡ʃ]}

\begin{figure}[H]
\includegraphics{../images/staticpalatography_stop.png}
\end{figure}

\newpage

\begin{center}
\textbf{{\color{red}{\HUGE END OF EXAM}}}\\

\end{center}
\newpage

\begin{center}
\textbf{{\color{blue}{\HUGE START OF EXAM\\}}}

\textbf{{\color{blue}{\HUGE Student ID: 12377\\}}}

\textbf{{\color{blue}{\HUGE 4:40\\}}}

\end{center}
\newpage

{\large Question 1}\\

Topic: Other (pre-midterm)\\
Source: Homework 2, Question 2\\

What would this Klingon phrase below be in English? How do you know?\\

{[vɑdʒqʰoqʰvɑm]}

\begin{figure}[H]
\includegraphics{../images/klingon.png}
\end{figure}

\newpage

{\large Question 2}\\

Topic: Skewed Distributions\\
Source: Week 5 Handout, Question 5\\

Explain why looking for patterns with consonants and vowels is a more reasonable approach to pattern finding in this dataset than looking for patterns with respect to all of the individual sounds in Ukrainian.\\

\begin{figure}[H]
\includegraphics{../images/ukrainian.png}
\end{figure}

\newpage

\begin{center}
\textbf{{\color{red}{\HUGE END OF EXAM}}}\\

\end{center}
\newpage

\begin{center}
\textbf{{\color{blue}{\HUGE START OF EXAM\\}}}

\textbf{{\color{blue}{\HUGE Student ID: 36593\\}}}

\textbf{{\color{blue}{\HUGE 4:50\\}}}

\end{center}
\newpage

{\large Question 1}\\

Topic: Skewed Distributions\\
Source: Quiz 4, Question 2\\

L$_X$ (Language X) has three vowels, [i], [a], and [u]. L$_X$ has tri-syllabic roots. If L$_X$ does not allow non-identical high vowels to co-occur, which one of the following tri-syllabic vocalic sequences do you predict to be unattested in L$_X$? Explain why.\\

\begin{itemize} \item {[u...i...a]} \item {[a...i...a]} \item {[u...u...a]} \item {[a...i...i]} \end{itemize}


\newpage

{\large Question 2}\\

Topic: Articulatory Phonetics\\
Source: Week 3 Handout, Question 3\\

Explain why the additional vowel below either does or does not belong in the phonetic natural class defined by the original set of SNAE vowels.\\

Original set: {[u]}, {[ʊ]}, {[oʊ]}, {[ɔ]}

Addition: {[ɔɪ]}


\newpage

\begin{center}
\textbf{{\color{red}{\HUGE END OF EXAM}}}\\

\end{center}
\newpage

\end{document}

