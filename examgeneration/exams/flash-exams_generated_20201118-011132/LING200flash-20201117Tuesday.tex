% Ensure that you compile using XeLaTeX !!! PDFTex has problems with some of the packages used
\documentclass[12pt]{article}
\setlength\parindent{0pt}

\usepackage{parskip}
\usepackage[margin=0.5in]{geometry}
\usepackage{fullpage}
\usepackage{moresize}
\usepackage{graphicx}
\usepackage{caption}
\usepackage{subcaption}
\usepackage{float}
\usepackage{xcolor}
\usepackage{soul}
\usepackage{fontspec}
\setmainfont{Doulos SIL}

\begin{document}

\begin{center}
\textbf{{\color{violet}{\HUGE 20201117 Tuesday\\}}}

\textbf{{\color{violet}{\HUGE ALL EXAMS\\}}}

\end{center}
\newpage

\begin{center}
\textbf{{\color{blue}{\HUGE START OF EXAM\\}}}

\textbf{{\color{blue}{\HUGE Student ID: 42792\\}}}

\textbf{{\color{blue}{\HUGE 4:00\\}}}

\end{center}
\newpage

{\large Question 1}\\

Topic: Articulatory Phonetics\\
Source: Homework 1, Question 3(b)\\

Explain why this is or is not a complete phonetic natural class in standard North American English.\\

{[b]}, {[n]}, {[ɡ]}, {[ʒ]}, {[v]}


\newpage

{\large Question 2}\\

Topic: Skewed Distributions\\
Source: Week 5 Handout, Question 6\\

If I gave you a new word in Malto, [di\_\_u], would it be possible to predict whether it's [d] or [t] that goes in the blank? Explain why or why not.\\

\begin{figure}[H]
\includegraphics{../images/malto.png}
\end{figure}

\newpage

\begin{center}
\textbf{{\color{red}{\HUGE END OF EXAM}}}\\

\end{center}
\newpage

\begin{center}
\textbf{{\color{blue}{\HUGE START OF EXAM\\}}}

\textbf{{\color{blue}{\HUGE Student ID: 94549\\}}}

\textbf{{\color{blue}{\HUGE 4:10\\}}}

\end{center}
\newpage

{\large Question 1}\\

Topic: Articulatory Phonetics\\
Source: Week 3 Discussion\\

Assuming a Standard North American English inventory, does this vowel need to have tenseness specified if you're giving a prose description? Why or why not?\\

{[ɔ]}


\newpage

{\large Question 2}\\

Topic: Skewed Distributions\\
Source: Week 5 Handout, Question 7\\

Explain how you would go about looking for co-occurrence restrictions in bi-syllabic signs in ASL. (Refer to the data that follows.)\\

\begin{figure}[H]
\includegraphics{../images/ASL_movement.png}
\end{figure}

\newpage

\begin{center}
\textbf{{\color{red}{\HUGE END OF EXAM}}}\\

\end{center}
\newpage

\begin{center}
\textbf{{\color{blue}{\HUGE START OF EXAM\\}}}

\textbf{{\color{blue}{\HUGE Student ID: 59669\\}}}

\textbf{{\color{blue}{\HUGE 4:20\\}}}

\end{center}
\newpage

{\large Question 1}\\

Topic: Acoustics\\
Source: Week 9 Handout, Question 7\\

Explain why each numbered, underlined statement is true or false. If it is false, explain one way that you could correct it.\\

We can look at the vertical location of the formants to determine something about the characteristics of individual speech sounds. For example, in the two spectrograms below, we can see that $^{22}$\ul{the first formant is higher in the spectrogram for sound 1 than it is for sound 2.} Because $^{23}$\ul{F1 is directly correlated with vowel height}, we know that $^{24}$\ul{the vowel pictured in sound 1 is a higher vowel than the one in sound 2}. For example, $^{25}$\ul{sound 1 might be an {[ɑ]} while sound 2 might be an {[i]}.}

\begin{figure}[H]
\includegraphics{../images/sound1a_sound2i.png}
\end{figure}

\newpage

{\large Question 2}\\

Topic: Articulatory Phonetics\\
Source: Week 3 Handout, Question 3\\

Explain why the additional vowel below either does or does not belong in the phonetic natural class defined by the original set of SNAE vowels.\\

Original set: {[u]}, {[ʊ]}, {[oʊ]}, {[ɔ]}

Addition: {[ɔɪ]}


\newpage

\begin{center}
\textbf{{\color{red}{\HUGE END OF EXAM}}}\\

\end{center}
\newpage

\begin{center}
\textbf{{\color{blue}{\HUGE START OF EXAM\\}}}

\textbf{{\color{blue}{\HUGE Student ID: 67125\\}}}

\textbf{{\color{blue}{\HUGE 4:30\\}}}

\end{center}
\newpage

{\large Question 1}\\

Topic: Acoustics\\
Source: Quiz 7, Question 1\\

Explain what you see in the spectrogram that tells you about the properties of the sounds in the pictured word.\\

\begin{figure}[H]
\includegraphics{../images/spectrogram_shoe.png}
\end{figure}

\newpage

{\large Question 2}\\

Topic: Phonological Features\\
Source: Homework 2, Question 1\\

Explain which sound should be removed to make this a natural class (assuming SNAE, except that there are no diphthongs, no [ə] or [ʌ], no syllabic consonants, and no [w̥]), and what the minimum set of features would be to describe the resulting natural class.\\

{[i]}, {[ɪ]}, {[e]}, {[ɛ]}, {[æ]}, {[ɑ]}, {[ɔ]}, {[o]}, {[ʊ]}, {[u]}, {[ʒ]}, {[k]}, {[ɡ]}, {[ŋ]}, {[w]}


\newpage

\begin{center}
\textbf{{\color{red}{\HUGE END OF EXAM}}}\\

\end{center}
\newpage

\begin{center}
\textbf{{\color{blue}{\HUGE START OF EXAM\\}}}

\textbf{{\color{blue}{\HUGE Student ID: 96220\\}}}

\textbf{{\color{blue}{\HUGE 4:40\\}}}

\end{center}
\newpage

{\large Question 1}\\

Topic: Skewed Distributions\\
Source: Quiz 4, Question 2\\

L$_X$ (Language X) has three vowels, [i], [a], and [u]. L$_X$ has tri-syllabic roots. If L$_X$ does not allow non-identical high vowels to co-occur, which one of the following tri-syllabic vocalic sequences do you predict to be unattested in L$_X$? Explain why.\\

\begin{itemize} \item {[u...i...a]} \item {[a...i...a]} \item {[u...u...a]} \item {[a...i...i]} \end{itemize}


\newpage

{\large Question 2}\\

Topic: Other (pre-midterm)\\
Source: Week 5 \& 6 Handouts\\

Explain how you could analyze this dataset in terms of sequential patterns vs. paradigmatic patterns.\\

\begin{figure}[H]
\includegraphics{../images/ukrainian.png}
\end{figure}

\newpage

\begin{center}
\textbf{{\color{red}{\HUGE END OF EXAM}}}\\

\end{center}
\newpage

\begin{center}
\textbf{{\color{blue}{\HUGE START OF EXAM\\}}}

\textbf{{\color{blue}{\HUGE Student ID: 16922\\}}}

\textbf{{\color{blue}{\HUGE 4:50\\}}}

\end{center}
\newpage

{\large Question 1}\\

Topic: Phonological Features\\
Source: Week 4 Discussion\\

Explain why phonological features are used instead of phonetic characteristics in analyzing datasets.\\


\newpage

{\large Question 2}\\

Topic: Acoustics\\
Source: Week 9 Handout, Question 7\\

Explain why each numbered, underlined statement is true or false. If it is false, explain one way that you could correct it.\\

In speech, air is set in vibrating motion by the lungs, so the lungs are the source of most speech sounds. $^{14}$\ul{The basic rate of vibration is called the fundamental frequency}; $^{15}$\ul{the fundamental frequency is also known as the timbre of the voice.} In addition to the source, we can also talk about a filter: $^{16}$\ul{the vocal folds act as a filter to shape the air from the lungs into the sounds we hear as different.} $^{17}$\ul{The mouth and nose act as resonance chambers}, and these also affect the qualities of the sounds.


\newpage

\begin{center}
\textbf{{\color{red}{\HUGE END OF EXAM}}}\\

\end{center}
\newpage

\end{document}

