% Ensure that you compile using XeLaTeX !!! PDFTex has problems with some of the packages used
\documentclass[12pt]{article}
\setlength\parindent{0pt}

\usepackage{parskip}
\usepackage[margin=0.5in]{geometry}
\usepackage{fullpage}
\usepackage{moresize}
\usepackage{graphicx}
\usepackage{caption}
\usepackage{subcaption}
\usepackage{float}
\usepackage{xcolor}
\usepackage{soul}
\usepackage{fontspec}
\setmainfont{Doulos SIL}

\begin{document}

\begin{center}
\textbf{{\color{violet}{\HUGE Friday, 5 June 2020\\}}}

\textbf{{\color{violet}{\HUGE ALL EXAMS (with notes)\\}}}

\end{center}
\newpage

\begin{center}
\textbf{{\color{blue}{\HUGE START OF EXAM\\}}}

\textbf{{\color{blue}{\HUGE Student ID: 6745\\}}}

\textbf{{\color{blue}{\HUGE 11:30 - 11:45 AM\\}}}

\end{center}
\newpage

{\large Question 1}\\

Source: Quiz 3, Question 2\\

L$_X$ has tri-syllabic roots. If L$_X$ does not allow non-identical high vowels to co-occur, which one of the following tri-syllabic vocalic sequences do you predict to be unattested in L$_X$? Explain why.\\

\begin{itemize} \item {[u...i...a]} \item {[a...i...a]} \item {[u...u...a]} \item {[a...i...i]} \end{itemize}


~\\
INSTRUCTOR NOTES: [u...i...a]


\vfill
Excellent (3) ~~~ Good (2.2) ~~~ Fair (1.7) ~~~ Poor (0)
\newpage

{\large Question 2}\\

Source: Day 4 Handout, Question 3\\

Explain how you would figure out what the Luiseño form is for the morpheme whose meaning is given below.\\

‘walk’

\begin{figure}[H]
\includegraphics{../images/luiseno.png}
\end{figure}

~\\
INSTRUCTOR NOTES: ([wukála])


\vfill
Excellent (3) ~~~ Good (2.2) ~~~ Fair (1.7) ~~~ Poor (0)
\newpage

{\large Question 3}\\

Source: Day 6 Handout, Question 11\\

What do the two signs below tell you about the phonological status of \underline{handshape} in ASL, and why?\\

\begin{figure}[H]
\includegraphics{../images/asl_stay.png}
\caption{STAY}
\end{figure}
\begin{figure}[H]
\includegraphics{../images/asl_awkward.png}
\caption{AWKWARD}
\end{figure}

~\\
INSTRUCTOR NOTES: nothing, because both handshape and movement are different


\vfill
Excellent (3) ~~~ Good (2.2) ~~~ Fair (1.7) ~~~ Poor (0)
\newpage

{\large Question 4}\\

Source: Homework 1, Question 3(b)\\

Explain why this is or is not a complete natural class in standard North American English.\\

{[i]}, {[u]}, {[eɪ]}


~\\
INSTRUCTOR NOTES: no; [oʊ] missing for tense vowels


\vfill
Excellent (3) ~~~ Good (2.2) ~~~ Fair (1.7) ~~~ Poor (0)
\newpage

{\large Question 5}\\

Source: Day 2 Handout, Part I, Question 11\\

How would this word be transcribed?\\ Follow-up question: Why did you use symbol [X] instead of symbol [Y]?\\

<goat>


~\\
INSTRUCTOR NOTES: [ɡoʊt]


\vfill
Excellent (3) ~~~ Good (2.2) ~~~ Fair (1.7) ~~~ Poor (0)
\newpage

\begin{center}
\textbf{{\color{red}{\HUGE END OF EXAM}}}\\

\end{center}
\newpage

\begin{center}
\textbf{{\color{blue}{\HUGE START OF EXAM\\}}}

\textbf{{\color{blue}{\HUGE Student ID: 9303\\}}}

\textbf{{\color{blue}{\HUGE 11:45 AM - 12:00 noon\\}}}

\end{center}
\newpage

{\large Question 1}\\

Source: Quiz 3, Question 2\\

L$_X$ has tri-syllabic roots. If L$_X$ does not allow non-identical high vowels to co-occur, which one of the following tri-syllabic vocalic sequences do you predict to be unattested in L$_X$? Explain why.\\

\begin{itemize} \item {[u...i...a]} \item {[a...i...a]} \item {[u...u...a]} \item {[a...i...i]} \end{itemize}


~\\
INSTRUCTOR NOTES: [u...i...a]


\vfill
Excellent (3) ~~~ Good (2.2) ~~~ Fair (1.7) ~~~ Poor (0)
\newpage

{\large Question 2}\\

Source: Day 2 Handout, Part II, Question 9\\

Explain how to figure out what the sound being produced is in this diagram.\\

\begin{figure}[H]
\includegraphics{../images/sagittal_z.png}
\end{figure}

~\\
INSTRUCTOR NOTES: [z] (check voicing, place, manner, and velum)


\vfill
Excellent (3) ~~~ Good (2.2) ~~~ Fair (1.7) ~~~ Poor (0)
\newpage

{\large Question 3}\\

Source: Day 6 Handout, Question 11\\

What do the two signs below tell you about the phonological status of \underline{handshape} in ASL, and why?\\

\begin{figure}[H]
\includegraphics{../images/asl_apple.png}
\caption{APPLE}
\end{figure}
\begin{figure}[H]
\includegraphics{../images/asl_candy.png}
\caption{CANDY}
\end{figure}

~\\
INSTRUCTOR NOTES: shows contrast because movement and location are same


\vfill
Excellent (3) ~~~ Good (2.2) ~~~ Fair (1.7) ~~~ Poor (0)
\newpage

{\large Question 4}\\

Source: Day 2 Handout, Part I, Question 11\\

How would this word be transcribed?\\ Follow-up question: Why did you use symbol [X] instead of symbol [Y]?\\

<square>


~\\
INSTRUCTOR NOTES: [skweɪɹ]


\vfill
Excellent (3) ~~~ Good (2.2) ~~~ Fair (1.7) ~~~ Poor (0)
\newpage

{\large Question 5}\\

Source: Day 2 Handout, Part I, Question 11\\

How would this word be transcribed?\\ Follow-up question: Why did you use symbol [X] instead of symbol [Y]?\\

<toy>


~\\
INSTRUCTOR NOTES: [tɔɪ]


\vfill
Excellent (3) ~~~ Good (2.2) ~~~ Fair (1.7) ~~~ Poor (0)
\newpage

\begin{center}
\textbf{{\color{red}{\HUGE END OF EXAM}}}\\

\end{center}
\newpage

\begin{center}
\textbf{{\color{blue}{\HUGE START OF EXAM\\}}}

\textbf{{\color{blue}{\HUGE Student ID: 8079\\}}}

\textbf{{\color{blue}{\HUGE 12:00 noon - 12:15 PM\\}}}

\end{center}
\newpage

{\large Question 1}\\

Source: Quiz 3, Question 1\\

L$_X$ (Language X) has three vowels, [i], [a], and [u]. It has bi-syllabic roots like Kikuyu. It does not allow non-identical high vowels to co-occur. Of the following nine logically possible vocalic sequences, which ones should be unattested in L$_X$? Explain why.\\

\begin{itemize} \item {[i...i]} \item {[i...a]} \item {[i...u]} \item {[a...i]} \item {[a...a]} \item {[a...u]} \item {[u...i]} \item {[u...a]} \item {[u...u]} \end{itemize}


~\\
INSTRUCTOR NOTES: [i...u], [u...i]


\vfill
Excellent (3) ~~~ Good (2.2) ~~~ Fair (1.7) ~~~ Poor (0)
\newpage

{\large Question 2}\\

Source: Day 2 Handout, Part I, Question 11\\

How would this word be transcribed?\\ Follow-up question: Why did you use symbol [X] instead of symbol [Y]?\\

<vacuum>


~\\
INSTRUCTOR NOTES: [vækjum]


\vfill
Excellent (3) ~~~ Good (2.2) ~~~ Fair (1.7) ~~~ Poor (0)
\newpage

{\large Question 3}\\

Source: Homework 1, Question 3(b)\\

Explain why this is or is not a complete natural class in standard North American English.\\

{[j]}, {[w]}


~\\
INSTRUCTOR NOTES: yes for voiced glides; [w̥] missing for glides


\vfill
Excellent (3) ~~~ Good (2.2) ~~~ Fair (1.7) ~~~ Poor (0)
\newpage

{\large Question 4}\\

Source: Day 2 Discussion\\

Assuming a Standard North American English inventory, does this vowel need to have tenseness specified if you're giving a prose description? Why or why not?\\

{[i]}


~\\
INSTRUCTOR NOTES: yes


\vfill
Excellent (3) ~~~ Good (2.2) ~~~ Fair (1.7) ~~~ Poor (0)
\newpage

{\large Question 5}\\

Source: Day 7 Handout, Question 9\\

What is the basic analysis of vowel length in this dataset, and what are the key pieces of evidence?\\

\begin{figure}[H]
\includegraphics{../images/malayalam.png}
\end{figure}

~\\
INSTRUCTOR NOTES: Short and long vowels appear to be contrastive (phonemic) in Malayalam, as evidenced by minimal pairs that differ only in terms of their vowel length, such as [koʈːa] ‘basket’ vs. [koːʈːa] ‘castle’ or [keʈːu] ‘burnt out’ vs. [keːʈːu] ‘heard.’


\vfill
Excellent (3) ~~~ Good (2.2) ~~~ Fair (1.7) ~~~ Poor (0)
\newpage

\begin{center}
\textbf{{\color{red}{\HUGE END OF EXAM}}}\\

\end{center}
\newpage

\begin{center}
\textbf{{\color{blue}{\HUGE START OF EXAM\\}}}

\textbf{{\color{blue}{\HUGE Student ID: 1794\\}}}

\textbf{{\color{blue}{\HUGE 12:15 PM - 12:30 PM\\}}}

\end{center}
\newpage

{\large Question 1}\\

Source: Quiz 3, Question 2\\

L$_X$ has tri-syllabic roots. If L$_X$ does not allow non-identical high vowels to co-occur, which one of the following tri-syllabic vocalic sequences do you predict to be unattested in L$_X$? Explain why.\\

\begin{itemize} \item {[u...i...a]} \item {[a...i...a]} \item {[u...u...a]} \item {[a...i...i]} \end{itemize}


~\\
INSTRUCTOR NOTES: [u...i...a]


\vfill
Excellent (3) ~~~ Good (2.2) ~~~ Fair (1.7) ~~~ Poor (0)
\newpage

{\large Question 2}\\

Source: Day 2 Handout, Part I, Question 11\\

How would this word be transcribed?\\ Follow-up question: Why did you use symbol [X] instead of symbol [Y]?\\

<segment>


~\\
INSTRUCTOR NOTES: [sɛɡmɛnt]


\vfill
Excellent (3) ~~~ Good (2.2) ~~~ Fair (1.7) ~~~ Poor (0)
\newpage

{\large Question 3}\\

Source: Day 7 Handout, Question 12\\

What is the basic analysis of oral and nasal vowels in this dataset, and what are the key pieces of evidence?\\

\begin{figure}[H]
\includegraphics{../images/english12.png}
\end{figure}

~\\
INSTRUCTOR NOTES: The pairs of sounds [i] and [ĩ], and [u] and [ũ], are each allophonic and therefore allophones of the same phoneme in English (though the two pairs represent two contrastive phonemes in English). The sounds [i] and [ĩ] are in complementary distribution in English, with [ĩ] occurring before the sounds [m] and [n], (e.g., [ɡlĩm] ‘gleam’ and [klĩn] ‘clean’) and [i] occurring elsewhere (e.g., [lip] ‘leap’). Similarly, the sounds [u] and [ũ] are also in complementary distribution, with exactly the same conditioning environments: [ũ] occurs before [m] and [n] (e.g., [dũm] ‘doom’ and [dũn] ‘dune’), and [u] occurs elsewhere (e.g. [but] ‘boot’). Thus, within each pair, we treat the vowels as allophonic. 


\vfill
Excellent (3) ~~~ Good (2.2) ~~~ Fair (1.7) ~~~ Poor (0)
\newpage

{\large Question 4}\\

Source: Day 2 Handout, Part I, Question 11\\

How would this word be transcribed?\\ Follow-up question: Why did you use symbol [X] instead of symbol [Y]?\\

<nice>


~\\
INSTRUCTOR NOTES: [nɑɪs]


\vfill
Excellent (3) ~~~ Good (2.2) ~~~ Fair (1.7) ~~~ Poor (0)
\newpage

{\large Question 5}\\

Source: Homework 1, Question 3(b)\\

Explain why this is or is not a complete natural class in standard North American English.\\

{[j]}, {[w]}


~\\
INSTRUCTOR NOTES: yes for voiced glides; [w̥] missing for glides


\vfill
Excellent (3) ~~~ Good (2.2) ~~~ Fair (1.7) ~~~ Poor (0)
\newpage

\begin{center}
\textbf{{\color{red}{\HUGE END OF EXAM}}}\\

\end{center}
\newpage

\begin{center}
\textbf{{\color{blue}{\HUGE START OF EXAM\\}}}

\textbf{{\color{blue}{\HUGE Student ID: 2357\\}}}

\textbf{{\color{blue}{\HUGE 12:30 - 12:45 PM\\}}}

\end{center}
\newpage

{\large Question 1}\\

Source: Day 5 Handout, Question 5\\

How would you look for co-occurrence restrictions between [s] and the vowels that come after it in this dataset?\\

\begin{figure}[H]
\includegraphics{../images/ukrainian.png}
\end{figure}

~\\
INSTRUCTOR NOTES: 


\vfill
Excellent (3) ~~~ Good (2.2) ~~~ Fair (1.7) ~~~ Poor (0)
\newpage

{\large Question 2}\\

Source: Day 4 Handout, Question 2(iii)\\

Explain how you would figure out the meaning of this Swahili word.\\

{[umefika]}

\begin{figure}[H]
\includegraphics{../images/swahiliverbs.png}
\end{figure}

~\\
INSTRUCTOR NOTES: (you (sg.) have arrived)


\vfill
Excellent (3) ~~~ Good (2.2) ~~~ Fair (1.7) ~~~ Poor (0)
\newpage

{\large Question 3}\\

Source: Day 2 Handout, Part I, Question 11\\

How would this word be transcribed?\\ Follow-up question: Why did you use symbol [X] instead of symbol [Y]?\\

<goat>


~\\
INSTRUCTOR NOTES: [ɡoʊt]


\vfill
Excellent (3) ~~~ Good (2.2) ~~~ Fair (1.7) ~~~ Poor (0)
\newpage

{\large Question 4}\\

Source: Day 2 Handout, Part II, Question 9\\

Explain how to figure out what the sound being produced is in this diagram.\\

\begin{figure}[H]
\includegraphics{../images/sagittal_m.png}
\end{figure}

~\\
INSTRUCTOR NOTES: [m] (check voicing, place, manner, and velum)


\vfill
Excellent (3) ~~~ Good (2.2) ~~~ Fair (1.7) ~~~ Poor (0)
\newpage

{\large Question 5}\\

Source: Day 6 Handout, Question 11\\

What do the two signs below tell you about the phonological status of \underline{handshape} in ASL, and why?\\

\begin{figure}[H]
\includegraphics{../images/asl_stay.png}
\caption{STAY}
\end{figure}
\begin{figure}[H]
\includegraphics{../images/asl_awkward.png}
\caption{AWKWARD}
\end{figure}

~\\
INSTRUCTOR NOTES: nothing, because both handshape and movement are different


\vfill
Excellent (3) ~~~ Good (2.2) ~~~ Fair (1.7) ~~~ Poor (0)
\newpage

\begin{center}
\textbf{{\color{red}{\HUGE END OF EXAM}}}\\

\end{center}
\newpage

\begin{center}
\textbf{{\color{blue}{\HUGE START OF EXAM\\}}}

\textbf{{\color{blue}{\HUGE Student ID: 3773\\}}}

\textbf{{\color{blue}{\HUGE 12:45 - 1:00 PM\\}}}

\end{center}
\newpage

{\large Question 1}\\

Source: Quiz 3, Question 2\\

L$_X$ has tri-syllabic roots. If L$_X$ does not allow non-identical high vowels to co-occur, which one of the following tri-syllabic vocalic sequences do you predict to be unattested in L$_X$? Explain why.\\

\begin{itemize} \item {[u...i...a]} \item {[a...i...a]} \item {[u...u...a]} \item {[a...i...i]} \end{itemize}


~\\
INSTRUCTOR NOTES: [u...i...a]


\vfill
Excellent (3) ~~~ Good (2.2) ~~~ Fair (1.7) ~~~ Poor (0)
\newpage

{\large Question 2}\\

Source: Day 2 Handout, Part II, Question 7\\

Is the symbol given a reasonable way to transcribe any of the sounds described below? If so, which one? If not, why not?\\

{[θ]}

\begin{itemize} \item voiceless palatal affricate \item voiced velar nasal \item voiceless glottal fricative \item voiced labiodental fricative \item voiced interdental fricative \item voiced palatal fricative \end{itemize}


~\\
INSTRUCTOR NOTES: no (voiceless interdental fricative)


\vfill
Excellent (3) ~~~ Good (2.2) ~~~ Fair (1.7) ~~~ Poor (0)
\newpage

{\large Question 3}\\

Source: Day 6 Handout, Question 11\\

What do the two signs below tell you about the phonological status of \underline{handshape} in ASL, and why?\\

\begin{figure}[H]
\includegraphics{../images/asl_apple.png}
\caption{APPLE}
\end{figure}
\begin{figure}[H]
\includegraphics{../images/asl_candy.png}
\caption{CANDY}
\end{figure}

~\\
INSTRUCTOR NOTES: shows contrast because movement and location are same


\vfill
Excellent (3) ~~~ Good (2.2) ~~~ Fair (1.7) ~~~ Poor (0)
\newpage

{\large Question 4}\\

Source: Day 2 Handout, Part I, Question 11\\

How would this word be transcribed?\\ Follow-up question: Why did you use symbol [X] instead of symbol [Y]?\\

<vacuum>


~\\
INSTRUCTOR NOTES: [vækjum]


\vfill
Excellent (3) ~~~ Good (2.2) ~~~ Fair (1.7) ~~~ Poor (0)
\newpage

{\large Question 5}\\

Source: Day 7 Handout, Question 2\\

Explain whether the rule below would apply to the form shown, and if so, what the effect of the rule would be. Assume the vowel inventory [i], [ɪ], [e], [ɛ], [ɑ], [u], [ʊ], [o], [ɔ].\\

/emus/

{[high vowel]} →  {[unround, front]} / {[front vowel]} C$_0$ \_\_ 


~\\
INSTRUCTOR NOTES: applies; [emis]


\vfill
Excellent (3) ~~~ Good (2.2) ~~~ Fair (1.7) ~~~ Poor (0)
\newpage

\begin{center}
\textbf{{\color{red}{\HUGE END OF EXAM}}}\\

\end{center}
\newpage

\begin{center}
\textbf{{\color{blue}{\HUGE START OF EXAM\\}}}

\textbf{{\color{blue}{\HUGE Student ID: 8951\\}}}

\textbf{{\color{blue}{\HUGE 1:00 - 1:15 PM\\}}}

\end{center}
\newpage

{\large Question 1}\\

Source: Day 5 Handout, Question 5\\

How would you look for co-occurrence restrictions between [s] and the vowels that come after it in this dataset?\\

\begin{figure}[H]
\includegraphics{../images/ukrainian.png}
\end{figure}

~\\
INSTRUCTOR NOTES: 


\vfill
Excellent (3) ~~~ Good (2.2) ~~~ Fair (1.7) ~~~ Poor (0)
\newpage

{\large Question 2}\\

Source: Day 6 Handout, Question 11\\

What do the two signs below tell you about the phonological status of \underline{handshape} in ASL, and why?\\

\begin{figure}[H]
\includegraphics{../images/asl_apple.png}
\caption{APPLE}
\end{figure}
\begin{figure}[H]
\includegraphics{../images/asl_now.png}
\caption{NOW}
\end{figure}

~\\
INSTRUCTOR NOTES: nothing, because handshape and location and movement are all also different


\vfill
Excellent (3) ~~~ Good (2.2) ~~~ Fair (1.7) ~~~ Poor (0)
\newpage

{\large Question 3}\\

Source: Day 4 Discussion\\

Explain what we mean by saying that linguistic patterns are \underline{productive}.\\


~\\
INSTRUCTOR NOTES: 


\vfill
Excellent (3) ~~~ Good (2.2) ~~~ Fair (1.7) ~~~ Poor (0)
\newpage

{\large Question 4}\\

Source: Homework 1, Question 3(b)\\

Explain why this is or is not a complete natural class in standard North American English.\\

{[f]}, {[θ]}, {[z]}, {[h]}


~\\
INSTRUCTOR NOTES: no; several fricatives missing


\vfill
Excellent (3) ~~~ Good (2.2) ~~~ Fair (1.7) ~~~ Poor (0)
\newpage

{\large Question 5}\\

Source: Day 2 Handout, Part I, Question 11\\

How would this word be transcribed?\\ Follow-up question: Why did you use symbol [X] instead of symbol [Y]?\\

<little>


~\\
INSTRUCTOR NOTES: [lɪɾl̩]


\vfill
Excellent (3) ~~~ Good (2.2) ~~~ Fair (1.7) ~~~ Poor (0)
\newpage

\begin{center}
\textbf{{\color{red}{\HUGE END OF EXAM}}}\\

\end{center}
\newpage

\begin{center}
\textbf{{\color{blue}{\HUGE START OF EXAM\\}}}

\textbf{{\color{blue}{\HUGE Student ID: 7336\\}}}

\textbf{{\color{blue}{\HUGE 1:15 - 1:30 PM\\}}}

\end{center}
\newpage

{\large Question 1}\\

Source: Quiz 3, Question 2\\

L$_X$ has tri-syllabic roots. If L$_X$ does not allow non-identical high vowels to co-occur, which one of the following tri-syllabic vocalic sequences do you predict to be unattested in L$_X$? Explain why.\\

\begin{itemize} \item {[u...i...a]} \item {[a...i...a]} \item {[u...u...a]} \item {[a...i...i]} \end{itemize}


~\\
INSTRUCTOR NOTES: [u...i...a]


\vfill
Excellent (3) ~~~ Good (2.2) ~~~ Fair (1.7) ~~~ Poor (0)
\newpage

{\large Question 2}\\

Source: Day 2 Discussion\\

Assuming a Standard North American English inventory, does this vowel need to have tenseness specified if you're giving a prose description? Why or why not?\\

{[u]}


~\\
INSTRUCTOR NOTES: yes


\vfill
Excellent (3) ~~~ Good (2.2) ~~~ Fair (1.7) ~~~ Poor (0)
\newpage

{\large Question 3}\\

Source: Day 7 Handout, Question 9\\

What is the basic analysis of vowel length in this dataset, and what are the key pieces of evidence?\\

\begin{figure}[H]
\includegraphics{../images/malayalam.png}
\end{figure}

~\\
INSTRUCTOR NOTES: Short and long vowels appear to be contrastive (phonemic) in Malayalam, as evidenced by minimal pairs that differ only in terms of their vowel length, such as [koʈːa] ‘basket’ vs. [koːʈːa] ‘castle’ or [keʈːu] ‘burnt out’ vs. [keːʈːu] ‘heard.’


\vfill
Excellent (3) ~~~ Good (2.2) ~~~ Fair (1.7) ~~~ Poor (0)
\newpage

{\large Question 4}\\

Source: Quiz 1, Question 7\\

Is this sentence prescriptive or descriptive? Explain why.\\

In casual styles of speaking, English speakers frequently end sentences with prepositions, but ending sentences with prepositions is avoided in formal styles.


~\\
INSTRUCTOR NOTES: 


\vfill
Excellent (3) ~~~ Good (2.2) ~~~ Fair (1.7) ~~~ Poor (0)
\newpage

{\large Question 5}\\

Source: Day 2 Handout\\

Is this a reasonable transcription of this word? Explain why.\\

<health>: {[hɛlð]}


~\\
INSTRUCTOR NOTES: no, [θ]


\vfill
Excellent (3) ~~~ Good (2.2) ~~~ Fair (1.7) ~~~ Poor (0)
\newpage

\begin{center}
\textbf{{\color{red}{\HUGE END OF EXAM}}}\\

\end{center}
\newpage

\begin{center}
\textbf{{\color{blue}{\HUGE START OF EXAM\\}}}

\textbf{{\color{blue}{\HUGE Student ID: 1715\\}}}

\textbf{{\color{blue}{\HUGE 1:30 - 1:45 PM\\}}}

\end{center}
\newpage

{\large Question 1}\\

Source: Day 5 Handout, Question 3\\

What evidence is there that there is a pattern in these data, assuming that these are the only CV and VC sequences that occur in some language?\\

{[sa]}, {[ʃi]}, {[za]}, {[ʒi]}, {[as]}, {[iʃ]}, {[az]}, {[iʒ]}


~\\
INSTRUCTOR NOTES: (the palatal sounds occur with the high vowel, while the alveolar sounds occur with the low vowel)


\vfill
Excellent (3) ~~~ Good (2.2) ~~~ Fair (1.7) ~~~ Poor (0)
\newpage

{\large Question 2}\\

Source: Homework 1, Question 3(b)\\

Explain why this is or is not a complete natural class in standard North American English.\\

{[f]}, {[θ]}, {[z]}, {[h]}


~\\
INSTRUCTOR NOTES: no; several fricatives missing


\vfill
Excellent (3) ~~~ Good (2.2) ~~~ Fair (1.7) ~~~ Poor (0)
\newpage

{\large Question 3}\\

Source: Day 2 Handout, Part II, Question 9\\

Explain how to figure out what the sound being produced is in this diagram.\\

\begin{figure}[H]
\includegraphics{../images/sagittal_t.png}
\end{figure}

~\\
INSTRUCTOR NOTES: [t] (check voicing, place, manner, and velum)


\vfill
Excellent (3) ~~~ Good (2.2) ~~~ Fair (1.7) ~~~ Poor (0)
\newpage

{\large Question 4}\\

Source: Day 6 Handout, Question 11\\

What do the two signs below tell you about the phonological status of \underline{handshape} in ASL, and why?\\

\begin{figure}[H]
\includegraphics{../images/asl_apple.png}
\caption{APPLE}
\end{figure}
\begin{figure}[H]
\includegraphics{../images/asl_now.png}
\caption{NOW}
\end{figure}

~\\
INSTRUCTOR NOTES: nothing, because handshape and location and movement are all also different


\vfill
Excellent (3) ~~~ Good (2.2) ~~~ Fair (1.7) ~~~ Poor (0)
\newpage

{\large Question 5}\\

Source: Day 2 Handout, Part I, Question 11\\

How would this word be transcribed?\\ Follow-up question: Why did you use symbol [X] instead of symbol [Y]?\\

<cough>


~\\
INSTRUCTOR NOTES: [kɑf]


\vfill
Excellent (3) ~~~ Good (2.2) ~~~ Fair (1.7) ~~~ Poor (0)
\newpage

\begin{center}
\textbf{{\color{red}{\HUGE END OF EXAM}}}\\

\end{center}
\newpage

\begin{center}
\textbf{{\color{blue}{\HUGE START OF EXAM\\}}}

\textbf{{\color{blue}{\HUGE Student ID: 3288\\}}}

\textbf{{\color{blue}{\HUGE 1:45 - 2:00 PM\\}}}

\end{center}
\newpage

{\large Question 1}\\

Source: Quiz 3, Question 2\\

L$_X$ has tri-syllabic roots. If L$_X$ does not allow non-identical high vowels to co-occur, which one of the following tri-syllabic vocalic sequences do you predict to be unattested in L$_X$? Explain why.\\

\begin{itemize} \item {[u...i...a]} \item {[a...i...a]} \item {[u...u...a]} \item {[a...i...i]} \end{itemize}


~\\
INSTRUCTOR NOTES: [u...i...a]


\vfill
Excellent (3) ~~~ Good (2.2) ~~~ Fair (1.7) ~~~ Poor (0)
\newpage

{\large Question 2}\\

Source: Day 2 Handout, Part II, Question 7\\

Is the symbol given a reasonable way to transcribe any of the sounds described below? If so, which one? If not, why not?\\

{[v]}

\begin{itemize} \item voiceless palatal affricate \item voiced velar nasal \item voiceless glottal fricative \item voiced labiodental fricative \item voiced interdental fricative \item voiced palatal fricative \end{itemize}


~\\
INSTRUCTOR NOTES: yes (voiced labiodental fricative)


\vfill
Excellent (3) ~~~ Good (2.2) ~~~ Fair (1.7) ~~~ Poor (0)
\newpage

{\large Question 3}\\

Source: Day 7 Handout, Question 9\\

What is the basic analysis of vowel length in this dataset, and what are the key pieces of evidence?\\

\begin{figure}[H]
\includegraphics{../images/malayalam.png}
\end{figure}

~\\
INSTRUCTOR NOTES: Short and long vowels appear to be contrastive (phonemic) in Malayalam, as evidenced by minimal pairs that differ only in terms of their vowel length, such as [koʈːa] ‘basket’ vs. [koːʈːa] ‘castle’ or [keʈːu] ‘burnt out’ vs. [keːʈːu] ‘heard.’


\vfill
Excellent (3) ~~~ Good (2.2) ~~~ Fair (1.7) ~~~ Poor (0)
\newpage

{\large Question 4}\\

Source: Day 2 Handout, Part I, Question 11\\

How would this word be transcribed?\\ Follow-up question: Why did you use symbol [X] instead of symbol [Y]?\\

<goat>


~\\
INSTRUCTOR NOTES: [ɡoʊt]


\vfill
Excellent (3) ~~~ Good (2.2) ~~~ Fair (1.7) ~~~ Poor (0)
\newpage

{\large Question 5}\\

Source: Day 2 Handout, Part I, Question 11\\

How would this word be transcribed?\\ Follow-up question: Why did you use symbol [X] instead of symbol [Y]?\\

<finger>


~\\
INSTRUCTOR NOTES: [fɪŋɡɹ̩]


\vfill
Excellent (3) ~~~ Good (2.2) ~~~ Fair (1.7) ~~~ Poor (0)
\newpage

\begin{center}
\textbf{{\color{red}{\HUGE END OF EXAM}}}\\

\end{center}
\newpage

\begin{center}
\textbf{{\color{blue}{\HUGE START OF EXAM\\}}}

\textbf{{\color{blue}{\HUGE Student ID: 4656\\}}}

\textbf{{\color{blue}{\HUGE 2:00 - 2:15 PM\\}}}

\end{center}
\newpage

{\large Question 1}\\

Source: Quiz 3, Question 1\\

L$_X$ (Language X) has three vowels, [i], [a], and [u]. It has bi-syllabic roots like Kikuyu. It does not allow non-identical high vowels to co-occur. Of the following nine logically possible vocalic sequences, which ones should be unattested in L$_X$? Explain why.\\

\begin{itemize} \item {[i...i]} \item {[i...a]} \item {[i...u]} \item {[a...i]} \item {[a...a]} \item {[a...u]} \item {[u...i]} \item {[u...a]} \item {[u...u]} \end{itemize}


~\\
INSTRUCTOR NOTES: [i...u], [u...i]


\vfill
Excellent (3) ~~~ Good (2.2) ~~~ Fair (1.7) ~~~ Poor (0)
\newpage

{\large Question 2}\\

Source: Day 2 Handout, Part I, Question 11\\

How would this word be transcribed?\\ Follow-up question: Why did you use symbol [X] instead of symbol [Y]?\\

<cough>


~\\
INSTRUCTOR NOTES: [kɑf]


\vfill
Excellent (3) ~~~ Good (2.2) ~~~ Fair (1.7) ~~~ Poor (0)
\newpage

{\large Question 3}\\

Source: Day 2 Handout, Part II, Question 7\\

Is the symbol given a reasonable way to transcribe any of the sounds described below? If so, which one? If not, why not?\\

{[ʃ]}

\begin{itemize} \item voiceless palatal affricate \item voiced velar nasal \item voiceless glottal fricative \item voiced labiodental fricative \item voiced interdental fricative \item voiced palatal fricative \end{itemize}


~\\
INSTRUCTOR NOTES: no (voiceless palatal fricative)


\vfill
Excellent (3) ~~~ Good (2.2) ~~~ Fair (1.7) ~~~ Poor (0)
\newpage

{\large Question 4}\\

Source: Day 2 Handout, Part II, Question 7\\

Is the symbol given a reasonable way to transcribe any of the sounds described below? If so, which one? If not, why not?\\

{[v]}

\begin{itemize} \item voiceless palatal affricate \item voiced velar nasal \item voiceless glottal fricative \item voiced labiodental fricative \item voiced interdental fricative \item voiced palatal fricative \end{itemize}


~\\
INSTRUCTOR NOTES: yes (voiced labiodental fricative)


\vfill
Excellent (3) ~~~ Good (2.2) ~~~ Fair (1.7) ~~~ Poor (0)
\newpage

{\large Question 5}\\

Source: Day 6 Handout, Question 11\\

What do the two signs below tell you about the phonological status of \underline{handshape} in ASL, and why?\\

\begin{figure}[H]
\includegraphics{../images/asl_apple.png}
\caption{APPLE}
\end{figure}
\begin{figure}[H]
\includegraphics{../images/asl_candy.png}
\caption{CANDY}
\end{figure}

~\\
INSTRUCTOR NOTES: shows contrast because movement and location are same


\vfill
Excellent (3) ~~~ Good (2.2) ~~~ Fair (1.7) ~~~ Poor (0)
\newpage

\begin{center}
\textbf{{\color{red}{\HUGE END OF EXAM}}}\\

\end{center}
\newpage

\begin{center}
\textbf{{\color{blue}{\HUGE START OF EXAM\\}}}

\textbf{{\color{blue}{\HUGE Student ID: 3419\\}}}

\textbf{{\color{blue}{\HUGE 2:15 - 2:30 PM\\}}}

\end{center}
\newpage

{\large Question 1}\\

Source: Quiz 3, Question 2\\

L$_X$ has tri-syllabic roots. If L$_X$ does not allow non-identical high vowels to co-occur, which one of the following tri-syllabic vocalic sequences do you predict to be unattested in L$_X$? Explain why.\\

\begin{itemize} \item {[u...i...a]} \item {[a...i...a]} \item {[u...u...a]} \item {[a...i...i]} \end{itemize}


~\\
INSTRUCTOR NOTES: [u...i...a]


\vfill
Excellent (3) ~~~ Good (2.2) ~~~ Fair (1.7) ~~~ Poor (0)
\newpage

{\large Question 2}\\

Source: Day 2 Handout, Part I, Question 11\\

How would this word be transcribed?\\ Follow-up question: Why did you use symbol [X] instead of symbol [Y]?\\

<vacuum>


~\\
INSTRUCTOR NOTES: [vækjum]


\vfill
Excellent (3) ~~~ Good (2.2) ~~~ Fair (1.7) ~~~ Poor (0)
\newpage

{\large Question 3}\\

Source: Day 7 Handout, Question 12\\

What is the basic analysis of oral and nasal vowels in this dataset, and what are the key pieces of evidence?\\

\begin{figure}[H]
\includegraphics{../images/english12.png}
\end{figure}

~\\
INSTRUCTOR NOTES: The pairs of sounds [i] and [ĩ], and [u] and [ũ], are each allophonic and therefore allophones of the same phoneme in English (though the two pairs represent two contrastive phonemes in English). The sounds [i] and [ĩ] are in complementary distribution in English, with [ĩ] occurring before the sounds [m] and [n], (e.g., [ɡlĩm] ‘gleam’ and [klĩn] ‘clean’) and [i] occurring elsewhere (e.g., [lip] ‘leap’). Similarly, the sounds [u] and [ũ] are also in complementary distribution, with exactly the same conditioning environments: [ũ] occurs before [m] and [n] (e.g., [dũm] ‘doom’ and [dũn] ‘dune’), and [u] occurs elsewhere (e.g. [but] ‘boot’). Thus, within each pair, we treat the vowels as allophonic. 


\vfill
Excellent (3) ~~~ Good (2.2) ~~~ Fair (1.7) ~~~ Poor (0)
\newpage

{\large Question 4}\\

Source: Day 6 Handout, Question 7\\

Explain how you would determine the phonological relationship between these two sounds (given below) in this dataset.\\

{[k]} and {[ɡ]}

\begin{figure}[H]
\includegraphics{../images/canadianfrench.png}
\end{figure}

~\\
INSTRUCTOR NOTES: contrastive; NEAR minimal pair; [evOk] ‘evoque’ vs. [vOg] ‘fashion’


\vfill
Excellent (3) ~~~ Good (2.2) ~~~ Fair (1.7) ~~~ Poor (0)
\newpage

{\large Question 5}\\

Source: Day 2 Handout, Part II, Question 7\\

Is the symbol given a reasonable way to transcribe any of the sounds described below? If so, which one? If not, why not?\\

{[ʃ]}

\begin{itemize} \item voiceless palatal affricate \item voiced velar nasal \item voiceless glottal fricative \item voiced labiodental fricative \item voiced interdental fricative \item voiced palatal fricative \end{itemize}


~\\
INSTRUCTOR NOTES: no (voiceless palatal fricative)


\vfill
Excellent (3) ~~~ Good (2.2) ~~~ Fair (1.7) ~~~ Poor (0)
\newpage

\begin{center}
\textbf{{\color{red}{\HUGE END OF EXAM}}}\\

\end{center}
\newpage

\begin{center}
\textbf{{\color{blue}{\HUGE START OF EXAM\\}}}

\textbf{{\color{blue}{\HUGE Student ID: 6801\\}}}

\textbf{{\color{blue}{\HUGE 2:30 - 2:45 PM\\}}}

\end{center}
\newpage

{\large Question 1}\\

Source: Day 5 Handout, Question 3\\

What evidence is there that there is a pattern in these data, assuming that these are the only CV and VC sequences that occur in some language?\\

{[sa]}, {[ʃi]}, {[za]}, {[ʒi]}, {[as]}, {[iʃ]}, {[az]}, {[iʒ]}


~\\
INSTRUCTOR NOTES: (the palatal sounds occur with the high vowel, while the alveolar sounds occur with the low vowel)


\vfill
Excellent (3) ~~~ Good (2.2) ~~~ Fair (1.7) ~~~ Poor (0)
\newpage

{\large Question 2}\\

Source: Day 2 Discussion\\

Assuming a Standard North American English inventory, does this vowel need to have tenseness specified if you're giving a prose description? Why or why not?\\

{[ɛ]}


~\\
INSTRUCTOR NOTES: yes


\vfill
Excellent (3) ~~~ Good (2.2) ~~~ Fair (1.7) ~~~ Poor (0)
\newpage

{\large Question 3}\\

Source: Day 6 Handout, Question 11\\

What do the two signs below tell you about the phonological status of \underline{handshape} in ASL, and why?\\

\begin{figure}[H]
\includegraphics{../images/asl_apple.png}
\caption{APPLE}
\end{figure}
\begin{figure}[H]
\includegraphics{../images/asl_candy.png}
\caption{CANDY}
\end{figure}

~\\
INSTRUCTOR NOTES: shows contrast because movement and location are same


\vfill
Excellent (3) ~~~ Good (2.2) ~~~ Fair (1.7) ~~~ Poor (0)
\newpage

{\large Question 4}\\

Source: Day 2 Handout, Part I, Question 11\\

How would this word be transcribed?\\ Follow-up question: Why did you use symbol [X] instead of symbol [Y]?\\

<little>


~\\
INSTRUCTOR NOTES: [lɪɾl̩]


\vfill
Excellent (3) ~~~ Good (2.2) ~~~ Fair (1.7) ~~~ Poor (0)
\newpage

{\large Question 5}\\

Source: Day 4 Handout, Question 2(iii)\\

Explain how you would figure out the meaning of this Swahili word.\\

{[watanipiɡa]}

\begin{figure}[H]
\includegraphics{../images/swahiliverbs.png}
\end{figure}

~\\
INSTRUCTOR NOTES: (they will beat me)


\vfill
Excellent (3) ~~~ Good (2.2) ~~~ Fair (1.7) ~~~ Poor (0)
\newpage

\begin{center}
\textbf{{\color{red}{\HUGE END OF EXAM}}}\\

\end{center}
\newpage

\begin{center}
\textbf{{\color{blue}{\HUGE START OF EXAM\\}}}

\textbf{{\color{blue}{\HUGE Student ID: 5581\\}}}

\textbf{{\color{blue}{\HUGE 2:45 - 3:00 PM\\}}}

\end{center}
\newpage

{\large Question 1}\\

Source: Day 5 Handout, Question 5\\

How would you look for co-occurrence restrictions between [s] and the vowels that come after it in this dataset?\\

\begin{figure}[H]
\includegraphics{../images/ukrainian.png}
\end{figure}

~\\
INSTRUCTOR NOTES: 


\vfill
Excellent (3) ~~~ Good (2.2) ~~~ Fair (1.7) ~~~ Poor (0)
\newpage

{\large Question 2}\\

Source: Day 2 Handout, Part I, Question 11\\

How would this word be transcribed?\\ Follow-up question: Why did you use symbol [X] instead of symbol [Y]?\\

<cough>


~\\
INSTRUCTOR NOTES: [kɑf]


\vfill
Excellent (3) ~~~ Good (2.2) ~~~ Fair (1.7) ~~~ Poor (0)
\newpage

{\large Question 3}\\

Source: Day 6 Handout, Question 11\\

What do the two signs below tell you about the phonological status of \underline{handshape} in ASL, and why?\\

\begin{figure}[H]
\includegraphics{../images/asl_stay.png}
\caption{STAY}
\end{figure}
\begin{figure}[H]
\includegraphics{../images/asl_awkward.png}
\caption{AWKWARD}
\end{figure}

~\\
INSTRUCTOR NOTES: nothing, because both handshape and movement are different


\vfill
Excellent (3) ~~~ Good (2.2) ~~~ Fair (1.7) ~~~ Poor (0)
\newpage

{\large Question 4}\\

Source: Day 7 Handout, Question 2\\

Explain whether the rule below would apply to the form shown, and if so, what the effect of the rule would be. Assume the vowel inventory [i], [ɪ], [e], [ɛ], [ɑ], [u], [ʊ], [o], [ɔ].\\

/emɛs/

{[high vowel]} →  {[unround, front]} / {[front vowel]} C$_0$ \_\_ 


~\\
INSTRUCTOR NOTES: doesn't apply


\vfill
Excellent (3) ~~~ Good (2.2) ~~~ Fair (1.7) ~~~ Poor (0)
\newpage

{\large Question 5}\\

Source: Day 2 Handout, Part II, Question 7\\

Is the symbol given a reasonable way to transcribe any of the sounds described below? If so, which one? If not, why not?\\

{[n]}

\begin{itemize} \item voiceless palatal affricate \item voiced velar nasal \item voiceless glottal fricative \item voiced labiodental fricative \item voiced interdental fricative \item voiced palatal fricative \end{itemize}


~\\
INSTRUCTOR NOTES: no (voiced alveolar nasal)


\vfill
Excellent (3) ~~~ Good (2.2) ~~~ Fair (1.7) ~~~ Poor (0)
\newpage

\begin{center}
\textbf{{\color{red}{\HUGE END OF EXAM}}}\\

\end{center}
\newpage

\begin{center}
\textbf{{\color{blue}{\HUGE START OF EXAM\\}}}

\textbf{{\color{blue}{\HUGE Student ID: 3420\\}}}

\textbf{{\color{blue}{\HUGE 3:00 - 3:15 PM\\}}}

\end{center}
\newpage

{\large Question 1}\\

Source: Day 5 Handout, Question 5\\

How would you look for co-occurrence restrictions between [s] and the vowels that come after it in this dataset?\\

\begin{figure}[H]
\includegraphics{../images/ukrainian.png}
\end{figure}

~\\
INSTRUCTOR NOTES: 


\vfill
Excellent (3) ~~~ Good (2.2) ~~~ Fair (1.7) ~~~ Poor (0)
\newpage

{\large Question 2}\\

Source: Day 2 Handout, Part II, Question 7\\

Is the symbol given a reasonable way to transcribe any of the sounds described below? If so, which one? If not, why not?\\

{[t͡ʃ]}

\begin{itemize} \item voiceless palatal affricate \item voiced velar nasal \item voiceless glottal fricative \item voiced labiodental fricative \item voiced interdental fricative \item voiced palatal fricative \end{itemize}


~\\
INSTRUCTOR NOTES: yes (voiceless palatal affricate)


\vfill
Excellent (3) ~~~ Good (2.2) ~~~ Fair (1.7) ~~~ Poor (0)
\newpage

{\large Question 3}\\

Source: Day 2 Handout\\

Is this a reasonable transcription of this word? Explain why.\\

<philosophy>: {[fəlɑsəfi]}


~\\
INSTRUCTOR NOTES: yes


\vfill
Excellent (3) ~~~ Good (2.2) ~~~ Fair (1.7) ~~~ Poor (0)
\newpage

{\large Question 4}\\

Source: Day 6 Handout, Question 11\\

What do the two signs below tell you about the phonological status of \underline{handshape} in ASL, and why?\\

\begin{figure}[H]
\includegraphics{../images/asl_stay.png}
\caption{STAY}
\end{figure}
\begin{figure}[H]
\includegraphics{../images/asl_awkward.png}
\caption{AWKWARD}
\end{figure}

~\\
INSTRUCTOR NOTES: nothing, because both handshape and movement are different


\vfill
Excellent (3) ~~~ Good (2.2) ~~~ Fair (1.7) ~~~ Poor (0)
\newpage

{\large Question 5}\\

Source: Day 2 Handout, Part I, Question 3\\

Explain why people might legitimately disagree about how many sounds this particular word contains.\\

<rice>


~\\
INSTRUCTOR NOTES: 


\vfill
Excellent (3) ~~~ Good (2.2) ~~~ Fair (1.7) ~~~ Poor (0)
\newpage

\begin{center}
\textbf{{\color{red}{\HUGE END OF EXAM}}}\\

\end{center}
\newpage

\begin{center}
\textbf{{\color{blue}{\HUGE START OF EXAM\\}}}

\textbf{{\color{blue}{\HUGE Student ID: 6427\\}}}

\textbf{{\color{blue}{\HUGE 3:15 - 3:30 PM\\}}}

\end{center}
\newpage

{\large Question 1}\\

Source: Day 5 Handout, Question 3\\

What evidence is there that there is a pattern in these data, assuming that these are the only CV and VC sequences that occur in some language?\\

{[sa]}, {[ʃi]}, {[za]}, {[ʒi]}, {[as]}, {[iʃ]}, {[az]}, {[iʒ]}


~\\
INSTRUCTOR NOTES: (the palatal sounds occur with the high vowel, while the alveolar sounds occur with the low vowel)


\vfill
Excellent (3) ~~~ Good (2.2) ~~~ Fair (1.7) ~~~ Poor (0)
\newpage

{\large Question 2}\\

Source: Day 2 Handout, Part I, Question 11\\

How would this word be transcribed?\\ Follow-up question: Why did you use symbol [X] instead of symbol [Y]?\\

<bird>


~\\
INSTRUCTOR NOTES: [bɹ̩d]


\vfill
Excellent (3) ~~~ Good (2.2) ~~~ Fair (1.7) ~~~ Poor (0)
\newpage

{\large Question 3}\\

Source: Day 2 Handout, Part I, Question 11\\

How would this word be transcribed?\\ Follow-up question: Why did you use symbol [X] instead of symbol [Y]?\\

<wealth>


~\\
INSTRUCTOR NOTES: [wɛlθ]


\vfill
Excellent (3) ~~~ Good (2.2) ~~~ Fair (1.7) ~~~ Poor (0)
\newpage

{\large Question 4}\\

Source: Day 2 Handout, Part II, Question 7\\

Is the symbol given a reasonable way to transcribe any of the sounds described below? If so, which one? If not, why not?\\

{[v]}

\begin{itemize} \item voiceless palatal affricate \item voiced velar nasal \item voiceless glottal fricative \item voiced labiodental fricative \item voiced interdental fricative \item voiced palatal fricative \end{itemize}


~\\
INSTRUCTOR NOTES: yes (voiced labiodental fricative)


\vfill
Excellent (3) ~~~ Good (2.2) ~~~ Fair (1.7) ~~~ Poor (0)
\newpage

{\large Question 5}\\

Source: Day 6 Handout, Question 11\\

What do the two signs below tell you about the phonological status of \underline{handshape} in ASL, and why?\\

\begin{figure}[H]
\includegraphics{../images/asl_stay.png}
\caption{STAY}
\end{figure}
\begin{figure}[H]
\includegraphics{../images/asl_awkward.png}
\caption{AWKWARD}
\end{figure}

~\\
INSTRUCTOR NOTES: nothing, because both handshape and movement are different


\vfill
Excellent (3) ~~~ Good (2.2) ~~~ Fair (1.7) ~~~ Poor (0)
\newpage

\begin{center}
\textbf{{\color{red}{\HUGE END OF EXAM}}}\\

\end{center}
\newpage

\begin{center}
\textbf{{\color{blue}{\HUGE START OF EXAM\\}}}

\textbf{{\color{blue}{\HUGE Student ID: 1956\\}}}

\textbf{{\color{blue}{\HUGE 3:30 - 3:45 PM\\}}}

\end{center}
\newpage

{\large Question 1}\\

Source: Day 5 Handout, Question 5\\

How would you look for co-occurrence restrictions between [s] and the vowels that come after it in this dataset?\\

\begin{figure}[H]
\includegraphics{../images/ukrainian.png}
\end{figure}

~\\
INSTRUCTOR NOTES: 


\vfill
Excellent (3) ~~~ Good (2.2) ~~~ Fair (1.7) ~~~ Poor (0)
\newpage

{\large Question 2}\\

Source: Day 2 Discussion\\

Assuming a Standard North American English inventory, does this vowel need to have tenseness specified if you're giving a prose description? Why or why not?\\

{[ɔ]}


~\\
INSTRUCTOR NOTES: yes


\vfill
Excellent (3) ~~~ Good (2.2) ~~~ Fair (1.7) ~~~ Poor (0)
\newpage

{\large Question 3}\\

Source: Day 2 Handout, Part I, Question 11\\

How would this word be transcribed?\\ Follow-up question: Why did you use symbol [X] instead of symbol [Y]?\\

<wealth>


~\\
INSTRUCTOR NOTES: [wɛlθ]


\vfill
Excellent (3) ~~~ Good (2.2) ~~~ Fair (1.7) ~~~ Poor (0)
\newpage

{\large Question 4}\\

Source: Day 7 Handout, Question 12\\

What is the basic analysis of oral and nasal vowels in this dataset, and what are the key pieces of evidence?\\

\begin{figure}[H]
\includegraphics{../images/english12.png}
\end{figure}

~\\
INSTRUCTOR NOTES: The pairs of sounds [i] and [ĩ], and [u] and [ũ], are each allophonic and therefore allophones of the same phoneme in English (though the two pairs represent two contrastive phonemes in English). The sounds [i] and [ĩ] are in complementary distribution in English, with [ĩ] occurring before the sounds [m] and [n], (e.g., [ɡlĩm] ‘gleam’ and [klĩn] ‘clean’) and [i] occurring elsewhere (e.g., [lip] ‘leap’). Similarly, the sounds [u] and [ũ] are also in complementary distribution, with exactly the same conditioning environments: [ũ] occurs before [m] and [n] (e.g., [dũm] ‘doom’ and [dũn] ‘dune’), and [u] occurs elsewhere (e.g. [but] ‘boot’). Thus, within each pair, we treat the vowels as allophonic. 


\vfill
Excellent (3) ~~~ Good (2.2) ~~~ Fair (1.7) ~~~ Poor (0)
\newpage

{\large Question 5}\\

Source: Day 2 Discussion\\

Assuming a Standard North American English inventory, does this vowel need to have tenseness specified if you're giving a prose description? Why or why not?\\

{[i]}


~\\
INSTRUCTOR NOTES: yes


\vfill
Excellent (3) ~~~ Good (2.2) ~~~ Fair (1.7) ~~~ Poor (0)
\newpage

\begin{center}
\textbf{{\color{red}{\HUGE END OF EXAM}}}\\

\end{center}
\newpage

\begin{center}
\textbf{{\color{blue}{\HUGE START OF EXAM\\}}}

\textbf{{\color{blue}{\HUGE Student ID: 5540\\}}}

\textbf{{\color{blue}{\HUGE 3:45 - 4:00 PM\\}}}

\end{center}
\newpage

{\large Question 1}\\

Source: Day 5 Handout, Question 5\\

How would you look for co-occurrence restrictions between [s] and the vowels that come after it in this dataset?\\

\begin{figure}[H]
\includegraphics{../images/ukrainian.png}
\end{figure}

~\\
INSTRUCTOR NOTES: 


\vfill
Excellent (3) ~~~ Good (2.2) ~~~ Fair (1.7) ~~~ Poor (0)
\newpage

{\large Question 2}\\

Source: Day 2 Handout, Part I, Question 11\\

How would this word be transcribed?\\ Follow-up question: Why did you use symbol [X] instead of symbol [Y]?\\

<bird>


~\\
INSTRUCTOR NOTES: [bɹ̩d]


\vfill
Excellent (3) ~~~ Good (2.2) ~~~ Fair (1.7) ~~~ Poor (0)
\newpage

{\large Question 3}\\

Source: Quiz 4, Question 5\\

What phonological relationships does this example show among the sounds [m], [n], and [ŋ], and why?\\

\begin{figure}[H]
\includegraphics{../images/quiz4question5_d.png}
\end{figure}

~\\
INSTRUCTOR NOTES: contrast (with a few neutralizations)


\vfill
Excellent (3) ~~~ Good (2.2) ~~~ Fair (1.7) ~~~ Poor (0)
\newpage

{\large Question 4}\\

Source: Day 2 Handout, Part II, Question 9\\

Explain how to figure out what the sound being produced is in this diagram.\\

\begin{figure}[H]
\includegraphics{../images/sagittal_p.png}
\end{figure}

~\\
INSTRUCTOR NOTES: [p] (check voicing, place, manner, and velum)


\vfill
Excellent (3) ~~~ Good (2.2) ~~~ Fair (1.7) ~~~ Poor (0)
\newpage

{\large Question 5}\\

Source: Day 6 Handout, Question 11\\

What do the two signs below tell you about the phonological status of \underline{handshape} in ASL, and why?\\

\begin{figure}[H]
\includegraphics{../images/asl_stay.png}
\caption{STAY}
\end{figure}
\begin{figure}[H]
\includegraphics{../images/asl_awkward.png}
\caption{AWKWARD}
\end{figure}

~\\
INSTRUCTOR NOTES: nothing, because both handshape and movement are different


\vfill
Excellent (3) ~~~ Good (2.2) ~~~ Fair (1.7) ~~~ Poor (0)
\newpage

\begin{center}
\textbf{{\color{red}{\HUGE END OF EXAM}}}\\

\end{center}
\newpage

\begin{center}
\textbf{{\color{blue}{\HUGE START OF EXAM\\}}}

\textbf{{\color{blue}{\HUGE Student ID: 4066\\}}}

\textbf{{\color{blue}{\HUGE 4:00 - 4:15 PM\\}}}

\end{center}
\newpage

{\large Question 1}\\

Source: Quiz 3, Question 1\\

L$_X$ (Language X) has three vowels, [i], [a], and [u]. It has bi-syllabic roots like Kikuyu. It does not allow non-identical high vowels to co-occur. Of the following nine logically possible vocalic sequences, which ones should be unattested in L$_X$? Explain why.\\

\begin{itemize} \item {[i...i]} \item {[i...a]} \item {[i...u]} \item {[a...i]} \item {[a...a]} \item {[a...u]} \item {[u...i]} \item {[u...a]} \item {[u...u]} \end{itemize}


~\\
INSTRUCTOR NOTES: [i...u], [u...i]


\vfill
Excellent (3) ~~~ Good (2.2) ~~~ Fair (1.7) ~~~ Poor (0)
\newpage

{\large Question 2}\\

Source: Quiz 2, Question 11\\

Does the morpheme ‘eye’ occur in this word? Why or why not?\\

<eyeglasses>


~\\
INSTRUCTOR NOTES: 


\vfill
Excellent (3) ~~~ Good (2.2) ~~~ Fair (1.7) ~~~ Poor (0)
\newpage

{\large Question 3}\\

Source: Day 2 Handout, Part I, Question 11\\

How would this word be transcribed?\\ Follow-up question: Why did you use symbol [X] instead of symbol [Y]?\\

<finger>


~\\
INSTRUCTOR NOTES: [fɪŋɡɹ̩]


\vfill
Excellent (3) ~~~ Good (2.2) ~~~ Fair (1.7) ~~~ Poor (0)
\newpage

{\large Question 4}\\

Source: Homework 1, Question 3(a)\\

Could this image be the result of producing the sound represented by the given IPA symbol? Why or why not?\\

{[d]}

\begin{figure}[H]
\includegraphics{../images/staticpalatography_fricative.png}
\end{figure}

~\\
INSTRUCTOR NOTES: no; space


\vfill
Excellent (3) ~~~ Good (2.2) ~~~ Fair (1.7) ~~~ Poor (0)
\newpage

{\large Question 5}\\

Source: Day 6 Handout, Question 11\\

What do the two signs below tell you about the phonological status of \underline{handshape} in ASL, and why?\\

\begin{figure}[H]
\includegraphics{../images/asl_apple.png}
\caption{APPLE}
\end{figure}
\begin{figure}[H]
\includegraphics{../images/asl_now.png}
\caption{NOW}
\end{figure}

~\\
INSTRUCTOR NOTES: nothing, because handshape and location and movement are all also different


\vfill
Excellent (3) ~~~ Good (2.2) ~~~ Fair (1.7) ~~~ Poor (0)
\newpage

\begin{center}
\textbf{{\color{red}{\HUGE END OF EXAM}}}\\

\end{center}
\newpage

\begin{center}
\textbf{{\color{blue}{\HUGE START OF EXAM\\}}}

\textbf{{\color{blue}{\HUGE Student ID: 9450\\}}}

\textbf{{\color{blue}{\HUGE 4:15 - 4:30 PM\\}}}

\end{center}
\newpage

{\large Question 1}\\

Source: Quiz 3, Question 1\\

L$_X$ (Language X) has three vowels, [i], [a], and [u]. It has bi-syllabic roots like Kikuyu. It does not allow non-identical high vowels to co-occur. Of the following nine logically possible vocalic sequences, which ones should be unattested in L$_X$? Explain why.\\

\begin{itemize} \item {[i...i]} \item {[i...a]} \item {[i...u]} \item {[a...i]} \item {[a...a]} \item {[a...u]} \item {[u...i]} \item {[u...a]} \item {[u...u]} \end{itemize}


~\\
INSTRUCTOR NOTES: [i...u], [u...i]


\vfill
Excellent (3) ~~~ Good (2.2) ~~~ Fair (1.7) ~~~ Poor (0)
\newpage

{\large Question 2}\\

Source: Day 2 Handout, Part II, Question 7\\

Is the symbol given a reasonable way to transcribe any of the sounds described below? If so, which one? If not, why not?\\

{[t͡ʃ]}

\begin{itemize} \item voiceless palatal affricate \item voiced velar nasal \item voiceless glottal fricative \item voiced labiodental fricative \item voiced interdental fricative \item voiced palatal fricative \end{itemize}


~\\
INSTRUCTOR NOTES: yes (voiceless palatal affricate)


\vfill
Excellent (3) ~~~ Good (2.2) ~~~ Fair (1.7) ~~~ Poor (0)
\newpage

{\large Question 3}\\

Source: Homework 1, Question 3(b)\\

Explain why this is or is not a complete natural class in standard North American English.\\

{[ɔ]}, {[ʊ]}, {[u]}, {[oʊ]}


~\\
INSTRUCTOR NOTES: yes (all back rounded vowels)


\vfill
Excellent (3) ~~~ Good (2.2) ~~~ Fair (1.7) ~~~ Poor (0)
\newpage

{\large Question 4}\\

Source: Day 2 Handout, Part I, Question 11\\

How would this word be transcribed?\\ Follow-up question: Why did you use symbol [X] instead of symbol [Y]?\\

<frog>


~\\
INSTRUCTOR NOTES: [fɹɑɡ]


\vfill
Excellent (3) ~~~ Good (2.2) ~~~ Fair (1.7) ~~~ Poor (0)
\newpage

{\large Question 5}\\

Source: Day 6 Handout, Question 11\\

What do the two signs below tell you about the phonological status of \underline{handshape} in ASL, and why?\\

\begin{figure}[H]
\includegraphics{../images/asl_stay.png}
\caption{STAY}
\end{figure}
\begin{figure}[H]
\includegraphics{../images/asl_awkward.png}
\caption{AWKWARD}
\end{figure}

~\\
INSTRUCTOR NOTES: nothing, because both handshape and movement are different


\vfill
Excellent (3) ~~~ Good (2.2) ~~~ Fair (1.7) ~~~ Poor (0)
\newpage

\begin{center}
\textbf{{\color{red}{\HUGE END OF EXAM}}}\\

\end{center}
\newpage

\begin{center}
\textbf{{\color{blue}{\HUGE START OF EXAM\\}}}

\textbf{{\color{blue}{\HUGE Student ID: 9918\\}}}

\textbf{{\color{blue}{\HUGE 4:30 - 4:45 PM\\}}}

\end{center}
\newpage

{\large Question 1}\\

Source: Quiz 3, Question 2\\

L$_X$ has tri-syllabic roots. If L$_X$ does not allow non-identical high vowels to co-occur, which one of the following tri-syllabic vocalic sequences do you predict to be unattested in L$_X$? Explain why.\\

\begin{itemize} \item {[u...i...a]} \item {[a...i...a]} \item {[u...u...a]} \item {[a...i...i]} \end{itemize}


~\\
INSTRUCTOR NOTES: [u...i...a]


\vfill
Excellent (3) ~~~ Good (2.2) ~~~ Fair (1.7) ~~~ Poor (0)
\newpage

{\large Question 2}\\

Source: Homework 1, Question 3(b)\\

Explain why this is or is not a complete natural class in standard North American English.\\

{[ɑ]}, {[u]}


~\\
INSTRUCTOR NOTES: no; several back vowels / back monophthongs missing


\vfill
Excellent (3) ~~~ Good (2.2) ~~~ Fair (1.7) ~~~ Poor (0)
\newpage

{\large Question 3}\\

Source: Day 2 Handout, Part I, Question 11\\

How would this word be transcribed?\\ Follow-up question: Why did you use symbol [X] instead of symbol [Y]?\\

<frog>


~\\
INSTRUCTOR NOTES: [fɹɑɡ]


\vfill
Excellent (3) ~~~ Good (2.2) ~~~ Fair (1.7) ~~~ Poor (0)
\newpage

{\large Question 4}\\

Source: Day 7 Handout, Question 9\\

What is the basic analysis of vowel length in this dataset, and what are the key pieces of evidence?\\

\begin{figure}[H]
\includegraphics{../images/malayalam.png}
\end{figure}

~\\
INSTRUCTOR NOTES: Short and long vowels appear to be contrastive (phonemic) in Malayalam, as evidenced by minimal pairs that differ only in terms of their vowel length, such as [koʈːa] ‘basket’ vs. [koːʈːa] ‘castle’ or [keʈːu] ‘burnt out’ vs. [keːʈːu] ‘heard.’


\vfill
Excellent (3) ~~~ Good (2.2) ~~~ Fair (1.7) ~~~ Poor (0)
\newpage

{\large Question 5}\\

Source: Day 2 Handout, Part II, Question 7\\

Is the symbol given a reasonable way to transcribe any of the sounds described below? If so, which one? If not, why not?\\

{[ʃ]}

\begin{itemize} \item voiceless palatal affricate \item voiced velar nasal \item voiceless glottal fricative \item voiced labiodental fricative \item voiced interdental fricative \item voiced palatal fricative \end{itemize}


~\\
INSTRUCTOR NOTES: no (voiceless palatal fricative)


\vfill
Excellent (3) ~~~ Good (2.2) ~~~ Fair (1.7) ~~~ Poor (0)
\newpage

\begin{center}
\textbf{{\color{red}{\HUGE END OF EXAM}}}\\

\end{center}
\newpage

\begin{center}
\textbf{{\color{blue}{\HUGE START OF EXAM\\}}}

\textbf{{\color{blue}{\HUGE Student ID: 6948\\}}}

\textbf{{\color{blue}{\HUGE 4:45 - 5:00 PM\\}}}

\end{center}
\newpage

{\large Question 1}\\

Source: Quiz 3, Question 2\\

L$_X$ has tri-syllabic roots. If L$_X$ does not allow non-identical high vowels to co-occur, which one of the following tri-syllabic vocalic sequences do you predict to be unattested in L$_X$? Explain why.\\

\begin{itemize} \item {[u...i...a]} \item {[a...i...a]} \item {[u...u...a]} \item {[a...i...i]} \end{itemize}


~\\
INSTRUCTOR NOTES: [u...i...a]


\vfill
Excellent (3) ~~~ Good (2.2) ~~~ Fair (1.7) ~~~ Poor (0)
\newpage

{\large Question 2}\\

Source: Day 2 Handout, Part I, Question 11\\

How would this word be transcribed?\\ Follow-up question: Why did you use symbol [X] instead of symbol [Y]?\\

<finger>


~\\
INSTRUCTOR NOTES: [fɪŋɡɹ̩]


\vfill
Excellent (3) ~~~ Good (2.2) ~~~ Fair (1.7) ~~~ Poor (0)
\newpage

{\large Question 3}\\

Source: Day 2 Handout, Part I, Question 11\\

How would this word be transcribed?\\ Follow-up question: Why did you use symbol [X] instead of symbol [Y]?\\

<frog>


~\\
INSTRUCTOR NOTES: [fɹɑɡ]


\vfill
Excellent (3) ~~~ Good (2.2) ~~~ Fair (1.7) ~~~ Poor (0)
\newpage

{\large Question 4}\\

Source: Day 6 Handout, Question 11\\

What do the two signs below tell you about the phonological status of \underline{handshape} in ASL, and why?\\

\begin{figure}[H]
\includegraphics{../images/asl_apple.png}
\caption{APPLE}
\end{figure}
\begin{figure}[H]
\includegraphics{../images/asl_now.png}
\caption{NOW}
\end{figure}

~\\
INSTRUCTOR NOTES: nothing, because handshape and location and movement are all also different


\vfill
Excellent (3) ~~~ Good (2.2) ~~~ Fair (1.7) ~~~ Poor (0)
\newpage

{\large Question 5}\\

Source: Homework 1, Question 3(b)\\

Explain why this is or is not a complete natural class in standard North American English.\\

{[ɔ]}, {[ʊ]}, {[u]}, {[oʊ]}


~\\
INSTRUCTOR NOTES: yes (all back rounded vowels)


\vfill
Excellent (3) ~~~ Good (2.2) ~~~ Fair (1.7) ~~~ Poor (0)
\newpage

\begin{center}
\textbf{{\color{red}{\HUGE END OF EXAM}}}\\

\end{center}
\newpage

\begin{center}
\textbf{{\color{blue}{\HUGE START OF EXAM\\}}}

\textbf{{\color{blue}{\HUGE Student ID: 3347\\}}}

\textbf{{\color{blue}{\HUGE 5:00 - 5:15 PM\\}}}

\end{center}
\newpage

{\large Question 1}\\

Source: Day 5 Handout, Question 5\\

How would you look for co-occurrence restrictions between [s] and the vowels that come after it in this dataset?\\

\begin{figure}[H]
\includegraphics{../images/ukrainian.png}
\end{figure}

~\\
INSTRUCTOR NOTES: 


\vfill
Excellent (3) ~~~ Good (2.2) ~~~ Fair (1.7) ~~~ Poor (0)
\newpage

{\large Question 2}\\

Source: Homework 2, Question 1\\

What would this Klingon phrase below be in English? How do you know?\\

{[pɑqqʰoqʰvetɬvo]}

\begin{figure}[H]
\includegraphics{../images/klingon.png}
\end{figure}

~\\
INSTRUCTOR NOTES: ‘from that so-called book’


\vfill
Excellent (3) ~~~ Good (2.2) ~~~ Fair (1.7) ~~~ Poor (0)
\newpage

{\large Question 3}\\

Source: Day 6 Handout, Question 11\\

What do the two signs below tell you about the phonological status of \underline{handshape} in ASL, and why?\\

\begin{figure}[H]
\includegraphics{../images/asl_apple.png}
\caption{APPLE}
\end{figure}
\begin{figure}[H]
\includegraphics{../images/asl_candy.png}
\caption{CANDY}
\end{figure}

~\\
INSTRUCTOR NOTES: shows contrast because movement and location are same


\vfill
Excellent (3) ~~~ Good (2.2) ~~~ Fair (1.7) ~~~ Poor (0)
\newpage

{\large Question 4}\\

Source: Day 2 Handout, Part I, Question 11\\

How would this word be transcribed?\\ Follow-up question: Why did you use symbol [X] instead of symbol [Y]?\\

<finger>


~\\
INSTRUCTOR NOTES: [fɪŋɡɹ̩]


\vfill
Excellent (3) ~~~ Good (2.2) ~~~ Fair (1.7) ~~~ Poor (0)
\newpage

{\large Question 5}\\

Source: Homework 1, Question 3(b)\\

Explain why this is or is not a complete natural class in standard North American English.\\

{[ɑ]}, {[u]}


~\\
INSTRUCTOR NOTES: no; several back vowels / back monophthongs missing


\vfill
Excellent (3) ~~~ Good (2.2) ~~~ Fair (1.7) ~~~ Poor (0)
\newpage

\begin{center}
\textbf{{\color{red}{\HUGE END OF EXAM}}}\\

\end{center}
\newpage

\begin{center}
\textbf{{\color{blue}{\HUGE START OF EXAM\\}}}

\textbf{{\color{blue}{\HUGE Student ID: 1887\\}}}

\textbf{{\color{blue}{\HUGE 5:15 - 5:30 PM\\}}}

\end{center}
\newpage

{\large Question 1}\\

Source: Quiz 3, Question 2\\

L$_X$ has tri-syllabic roots. If L$_X$ does not allow non-identical high vowels to co-occur, which one of the following tri-syllabic vocalic sequences do you predict to be unattested in L$_X$? Explain why.\\

\begin{itemize} \item {[u...i...a]} \item {[a...i...a]} \item {[u...u...a]} \item {[a...i...i]} \end{itemize}


~\\
INSTRUCTOR NOTES: [u...i...a]


\vfill
Excellent (3) ~~~ Good (2.2) ~~~ Fair (1.7) ~~~ Poor (0)
\newpage

{\large Question 2}\\

Source: Day 2 Handout, Part II, Question 13\\

Explain why this image does or does not match the description.\\

\begin{itemize} \item A one-handed sign. \item Location: In front of signer’s chin. \item Handshape: Starts with an “L” shape; proximal joint of index finger folds down during the sign. \item Movement: Hand starts on far side of signer’s body and moves horizontally straight across. \end{itemize}

\begin{figure}[H]
\includegraphics{../images/taiwansign_jealous.png}
\caption{JEALOUS}
\end{figure}

~\\
INSTRUCTOR NOTES: no; handshape and movement are wrong


\vfill
Excellent (3) ~~~ Good (2.2) ~~~ Fair (1.7) ~~~ Poor (0)
\newpage

{\large Question 3}\\

Source: Day 7 Handout, Question 9\\

What is the basic analysis of vowel length in this dataset, and what are the key pieces of evidence?\\

\begin{figure}[H]
\includegraphics{../images/malayalam.png}
\end{figure}

~\\
INSTRUCTOR NOTES: Short and long vowels appear to be contrastive (phonemic) in Malayalam, as evidenced by minimal pairs that differ only in terms of their vowel length, such as [koʈːa] ‘basket’ vs. [koːʈːa] ‘castle’ or [keʈːu] ‘burnt out’ vs. [keːʈːu] ‘heard.’


\vfill
Excellent (3) ~~~ Good (2.2) ~~~ Fair (1.7) ~~~ Poor (0)
\newpage

{\large Question 4}\\

Source: Day 2 Handout, Part I, Question 11\\

How would this word be transcribed?\\ Follow-up question: Why did you use symbol [X] instead of symbol [Y]?\\

<frog>


~\\
INSTRUCTOR NOTES: [fɹɑɡ]


\vfill
Excellent (3) ~~~ Good (2.2) ~~~ Fair (1.7) ~~~ Poor (0)
\newpage

{\large Question 5}\\

Source: Day 2 Handout, Part I, Question 11\\

How would this word be transcribed?\\ Follow-up question: Why did you use symbol [X] instead of symbol [Y]?\\

<little>


~\\
INSTRUCTOR NOTES: [lɪɾl̩]


\vfill
Excellent (3) ~~~ Good (2.2) ~~~ Fair (1.7) ~~~ Poor (0)
\newpage

\begin{center}
\textbf{{\color{red}{\HUGE END OF EXAM}}}\\

\end{center}
\newpage

\end{document}

