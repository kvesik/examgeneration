% Ensure that you compile using XeLaTeX !!! PDFTex has problems with some of the packages used
\documentclass[12pt]{article}
\setlength\parindent{0pt}

\usepackage{parskip}
\usepackage[margin=0.5in]{geometry}
\usepackage{fullpage}
\usepackage{moresize}
\usepackage{graphicx}
\usepackage{caption}
\usepackage{subcaption}
\usepackage{float}
\usepackage{xcolor}
\usepackage{soul}
\usepackage{fontspec}
\setmainfont{Doulos SIL}

\begin{document}

\begin{center}
\textbf{{\color{violet}{\HUGE 20201103 Tuesday\\}}}

\textbf{{\color{violet}{\HUGE ALL EXAMS (with notes)\\}}}

\end{center}
\newpage

\begin{center}
\textbf{{\color{blue}{\HUGE START OF EXAM\\}}}

\textbf{{\color{blue}{\HUGE Student ID: 43803\\}}}

\textbf{{\color{blue}{\HUGE 4:00\\}}}

\end{center}
\newpage

{\large Question 1}\\

Topic: Articulatory Phonetics\\
Source: Week 3 Handout, Question 3\\

Explain why the additional vowel below either does or does not belong in the phonetic natural class defined by the original set of SNAE vowels.\\

Original set: {[u]}, {[ʊ]}, {[oʊ]}, {[ɔ]}

Addition: {[ɔɪ]}


~\\
INSTRUCTOR NOTES: should recognize that there's more than one vowel sound, which makes it somewhat difficult to categorize; best answers will say that the diphthong is crucially a diphthong and so can't also go in this class


\vfill
Excellent (3) ~~~ Good (2.2) ~~~ Fair (1.7) ~~~ Poor (0)
\newpage

{\large Question 2}\\

Topic: Skewed Distributions\\
Source: Quiz 4, Question 3\\

L$_X$ (Language X) has three vowels, [i], [a], and [u]. L$_X$ has tetra-syllabic roots. If L$_X$ does not allow non-identical high vowels to co-occur, which one of the following tetra-syllabic vocalic sequences do you predict to be unattested in L$_X$? Explain why.\\

\begin{itemize} \item {[i...i...u...u]} \item {[a...a...i...i]} \item {[u...u...u...u]} \item {[i...i...i...a]} \end{itemize}


~\\
INSTRUCTOR NOTES: [i...i...u...u]


\vfill
Excellent (3) ~~~ Good (2.2) ~~~ Fair (1.7) ~~~ Poor (0)
\newpage

\begin{center}
\textbf{{\color{red}{\HUGE END OF EXAM}}}\\

\end{center}
\newpage

\begin{center}
\textbf{{\color{blue}{\HUGE START OF EXAM\\}}}

\textbf{{\color{blue}{\HUGE Student ID: 18870\\}}}

\textbf{{\color{blue}{\HUGE 4:10\\}}}

\end{center}
\newpage

{\large Question 1}\\

Topic: Skewed Distributions\\
Source: Week 5 Handout, Question 3\\

What evidence is there that there is a pattern in these data, assuming that these are the only CV and VC sequences that occur in some language?\\

{[sa]}, {[ʃi]}, {[za]}, {[ʒi]}, {[as]}, {[iʃ]}, {[az]}, {[iʒ]}


~\\
INSTRUCTOR NOTES: (the palatal sounds occur with the high vowel, while the alveolar sounds occur with the low vowel)


\vfill
Excellent (3) ~~~ Good (2.2) ~~~ Fair (1.7) ~~~ Poor (0)
\newpage

{\large Question 2}\\

Topic: Phonological Features\\
Source: Quiz 3, Question 12\\

Explain how you figure out which feature is involved in the process of umlaut shown below.\\

\begin{figure}[H]
\includegraphics{../images/dutch.png}
\end{figure}

~\\
INSTRUCTOR NOTES: we look to see which vowels are affected, and compare them to see which feature is DIFFERENT (not e.g. what features they share); so since the vowels in the singular and plural are identical except that the singular forms are back and the plural are front, it's the feature [back] that is relevant / changing / involved (not e.g. the feature [round] just because all of the vowels are round)


\vfill
Excellent (3) ~~~ Good (2.2) ~~~ Fair (1.7) ~~~ Poor (0)
\newpage

\begin{center}
\textbf{{\color{red}{\HUGE END OF EXAM}}}\\

\end{center}
\newpage

\begin{center}
\textbf{{\color{blue}{\HUGE START OF EXAM\\}}}

\textbf{{\color{blue}{\HUGE Student ID: empty\\}}}

\textbf{{\color{blue}{\HUGE 4:20\\}}}

\end{center}
\newpage

\begin{center}
\textbf{{\color{blue}{\HUGE START OF EXAM\\}}}

\textbf{{\color{blue}{\HUGE Student ID: 28926\\}}}

\textbf{{\color{blue}{\HUGE 4:30\\}}}

\end{center}
\newpage

{\large Question 1}\\

Topic: Phonological Features\\
Source: Homework 2, Question 1\\

Explain which sound should be removed to make this a natural class (assuming SNAE, except that there are no diphthongs, no [ə] or [ʌ], no syllabic consonants, and no [w̥]), and what the minimum set of features would be to describe the resulting natural class.\\

{[v]}, {[z]}, {[ʃ]}, {[ʒ]}, {[ð]}


~\\
INSTRUCTOR NOTES: [ʃ] should be removed, so that we have the natural class of voiced fricatives; this could be minimally represented with [+voice, +cont, -son]


\vfill
Excellent (3) ~~~ Good (2.2) ~~~ Fair (1.7) ~~~ Poor (0)
\newpage

{\large Question 2}\\

Topic: Articulatory Phonetics\\
Source: Homework 1, Question 3(a)\\

Could this image be the result of producing the sound represented by the given IPA symbol? Why or why not?\\

{[ɾ]}

\begin{figure}[H]
\includegraphics{../images/staticpalatography_stop.png}
\end{figure}

~\\
INSTRUCTOR NOTES: yes


\vfill
Excellent (3) ~~~ Good (2.2) ~~~ Fair (1.7) ~~~ Poor (0)
\newpage

\begin{center}
\textbf{{\color{red}{\HUGE END OF EXAM}}}\\

\end{center}
\newpage

\begin{center}
\textbf{{\color{blue}{\HUGE START OF EXAM\\}}}

\textbf{{\color{blue}{\HUGE Student ID: 83324\\}}}

\textbf{{\color{blue}{\HUGE 4:40\\}}}

\end{center}
\newpage

{\large Question 1}\\

Topic: Transcription\\
Source: Week 2 Handout, Part II\\

Is this a reasonable transcription of this word? Explain why.\\

<wimp>: {[wimp]}


~\\
INSTRUCTOR NOTES: no, [ɪ]


\vfill
Excellent (3) ~~~ Good (2.2) ~~~ Fair (1.7) ~~~ Poor (0)
\newpage

{\large Question 2}\\

Topic: Articulatory Phonetics\\
Source: Week 3 Handout, Question 9\\

Explain how to figure out what the sound being produced is in this diagram.\\

\begin{figure}[H]
\includegraphics{../images/sagittal_t.png}
\end{figure}

~\\
INSTRUCTOR NOTES: [t] (check voicing, place, manner, and velum)


\vfill
Excellent (3) ~~~ Good (2.2) ~~~ Fair (1.7) ~~~ Poor (0)
\newpage

\begin{center}
\textbf{{\color{red}{\HUGE END OF EXAM}}}\\

\end{center}
\newpage

\begin{center}
\textbf{{\color{blue}{\HUGE START OF EXAM\\}}}

\textbf{{\color{blue}{\HUGE Student ID: 80815\\}}}

\textbf{{\color{blue}{\HUGE 4:50\\}}}

\end{center}
\newpage

{\large Question 1}\\

Topic: Phonological Features\\
Source: Homework 2, Question 1\\

Explain which sound should be removed to make this a natural class (assuming SNAE, except that there are no diphthongs, no [ə] or [ʌ], no syllabic consonants, and no [w̥]), and what the minimum set of features would be to describe the resulting natural class.\\

{[i]}, {[ɪ]}, {[e]}, {[ɛ]}, {[æ]}, {[ɑ]}, {[ɔ]}, {[o]}, {[ʊ]}, {[u]}, {[ʒ]}, {[k]}, {[ɡ]}, {[ŋ]}, {[w]}


~\\
INSTRUCTOR NOTES: [ʒ] should be removed, so that we have the natural class of dorsal segments; this could be minimally represented with [DORSAL]


\vfill
Excellent (3) ~~~ Good (2.2) ~~~ Fair (1.7) ~~~ Poor (0)
\newpage

{\large Question 2}\\

Topic: Transcription\\
Source: Week 2 Handout, Part II\\

Is this a reasonable transcription of this word? Explain why.\\

<health>: {[hɛlð]}


~\\
INSTRUCTOR NOTES: no, [θ]


\vfill
Excellent (3) ~~~ Good (2.2) ~~~ Fair (1.7) ~~~ Poor (0)
\newpage

\begin{center}
\textbf{{\color{red}{\HUGE END OF EXAM}}}\\

\end{center}
\newpage

\end{document}

