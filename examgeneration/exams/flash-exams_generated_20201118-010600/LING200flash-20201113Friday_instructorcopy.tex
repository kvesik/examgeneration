% Ensure that you compile using XeLaTeX !!! PDFTex has problems with some of the packages used
\documentclass[12pt]{article}
\setlength\parindent{0pt}

\usepackage{parskip}
\usepackage[margin=0.5in]{geometry}
\usepackage{fullpage}
\usepackage{moresize}
\usepackage{graphicx}
\usepackage{caption}
\usepackage{subcaption}
\usepackage{float}
\usepackage{xcolor}
\usepackage{soul}
\usepackage{fontspec}
\setmainfont{Doulos SIL}

\begin{document}

\begin{center}
\textbf{{\color{violet}{\HUGE 20201113 Friday\\}}}

\textbf{{\color{violet}{\HUGE ALL EXAMS (with notes)\\}}}

\end{center}
\newpage

\begin{center}
\textbf{{\color{blue}{\HUGE START OF EXAM\\}}}

\textbf{{\color{blue}{\HUGE Student ID: 45248\\}}}

\textbf{{\color{blue}{\HUGE 4:00\\}}}

\end{center}
\newpage

{\large Question 1}\\

Topic: Other (pre-midterm)\\
Source: Week 5 \& 6 Handouts\\

Explain how you could analyze this dataset in terms of sequential patterns vs. paradigmatic patterns.\\

\begin{figure}[H]
\includegraphics{../images/ukrainian.png}
\end{figure}

~\\
INSTRUCTOR NOTES: 


\vfill
Excellent (3) ~~~ Good (2.2) ~~~ Fair (1.7) ~~~ Poor (0)
\newpage

{\large Question 2}\\

Topic: Phonological Features\\
Source: Homework 2, Question 1\\

Explain which sound should be removed to make this a natural class (assuming SNAE, except that there are no diphthongs, no [ə] or [ʌ], no syllabic consonants, and no [w̥]), and what the minimum set of features would be to describe the resulting natural class.\\

{[b]}, {[d]}, {[z]}, {[ɾ]}, {[n]}, {[l]}, {[ɹ]}


~\\
INSTRUCTOR NOTES: [b] should be removed, so that we have the natural class of voiced alveolars; this could be minimally represented with [+voice, -distributed]


\vfill
Excellent (3) ~~~ Good (2.2) ~~~ Fair (1.7) ~~~ Poor (0)
\newpage

\begin{center}
\textbf{{\color{red}{\HUGE END OF EXAM}}}\\

\end{center}
\newpage

\begin{center}
\textbf{{\color{blue}{\HUGE START OF EXAM\\}}}

\textbf{{\color{blue}{\HUGE Student ID: 99531\\}}}

\textbf{{\color{blue}{\HUGE 4:10\\}}}

\end{center}
\newpage

{\large Question 1}\\

Topic: Acoustics\\
Source: Week 9 Handout, Question 7\\

Explain why each numbered, underlined statement is true or false. If it is false, explain one way that you could correct it.\\

In speech, air is set in vibrating motion by the lungs, so the lungs are the source of most speech sounds. $^{14}$\ul{The basic rate of vibration is called the fundamental frequency}; $^{15}$\ul{the fundamental frequency is also known as the timbre of the voice.} In addition to the source, we can also talk about a filter: $^{16}$\ul{the vocal folds act as a filter to shape the air from the lungs into the sounds we hear as different.} $^{17}$\ul{The mouth and nose act as resonance chambers}, and these also affect the qualities of the sounds.


~\\
INSTRUCTOR NOTES: 14 - true.\\15 - false (the fundamental frequency is also known as the pitch of the voice, or F0).\\16 - false (the mouth and nose act as a filter to shape the air from the vocal folds in the sounds we hear as different, or vocal tract, or articulatory and resonance chambers, etc).\\17 - true.


\vfill
Excellent (3) ~~~ Good (2.2) ~~~ Fair (1.7) ~~~ Poor (0)
\newpage

{\large Question 2}\\

Topic: Transcription\\
Source: Week 2 Handout, Part II\\

Is this a reasonable transcription of this word? Explain why.\\

<wimp>: {[wimp]}


~\\
INSTRUCTOR NOTES: no, [ɪ]


\vfill
Excellent (3) ~~~ Good (2.2) ~~~ Fair (1.7) ~~~ Poor (0)
\newpage

\begin{center}
\textbf{{\color{red}{\HUGE END OF EXAM}}}\\

\end{center}
\newpage

\begin{center}
\textbf{{\color{blue}{\HUGE START OF EXAM\\}}}

\textbf{{\color{blue}{\HUGE Student ID: 51191\\}}}

\textbf{{\color{blue}{\HUGE 4:20\\}}}

\end{center}
\newpage

{\large Question 1}\\

Topic: Articulatory Phonetics\\
Source: Homework 1, Question 3(b)\\

Explain why this is or is not a complete phonetic natural class in standard North American English.\\

{[f]}, {[θ]}, {[z]}, {[h]}


~\\
INSTRUCTOR NOTES: no; several fricatives missing


\vfill
Excellent (3) ~~~ Good (2.2) ~~~ Fair (1.7) ~~~ Poor (0)
\newpage

{\large Question 2}\\

Topic: Other (pre-midterm)\\
Source: Week 5 \& 6 Handouts\\

Explain how you could analyze this dataset in terms of sequential patterns vs. paradigmatic patterns.\\

\begin{figure}[H]
\includegraphics{../images/ukrainian.png}
\end{figure}

~\\
INSTRUCTOR NOTES: 


\vfill
Excellent (3) ~~~ Good (2.2) ~~~ Fair (1.7) ~~~ Poor (0)
\newpage

\begin{center}
\textbf{{\color{red}{\HUGE END OF EXAM}}}\\

\end{center}
\newpage

\begin{center}
\textbf{{\color{blue}{\HUGE START OF EXAM\\}}}

\textbf{{\color{blue}{\HUGE Student ID: 11925\\}}}

\textbf{{\color{blue}{\HUGE 4:30\\}}}

\end{center}
\newpage

{\large Question 1}\\

Topic: Transcription\\
Source: Week 2 Handout, Part II, Question 11\\

How would this word be transcribed?\\ (Kathleen will then ask a follow-up question about your transcription.)\\

<finger>


~\\
INSTRUCTOR NOTES: [fɪŋɡɹ̩]


\vfill
Excellent (3) ~~~ Good (2.2) ~~~ Fair (1.7) ~~~ Poor (0)
\newpage

{\large Question 2}\\

Topic: Acoustics\\
Source: Quiz 7, Question 2\\

Explain what you see in the spectrogram that tells you about the properties of the sounds in the pictured word.\\

\begin{figure}[H]
\includegraphics{../images/spectrogram_hippo.png}
\end{figure}

~\\
INSTRUCTOR NOTES: hippo: stop in middle indicated by silence / white space; vowels on either side; V1 has low F1 and high F2 (=high front V) and V2 has higher F1 and lower F2 (=mid to low backer vowel)


\vfill
Excellent (3) ~~~ Good (2.2) ~~~ Fair (1.7) ~~~ Poor (0)
\newpage

\begin{center}
\textbf{{\color{red}{\HUGE END OF EXAM}}}\\

\end{center}
\newpage

\begin{center}
\textbf{{\color{blue}{\HUGE START OF EXAM\\}}}

\textbf{{\color{blue}{\HUGE Student ID: 43736\\}}}

\textbf{{\color{blue}{\HUGE 4:40\\}}}

\end{center}
\newpage

{\large Question 1}\\

Topic: Articulatory Phonetics\\
Source: Week 3 Handout, Question 3\\

Explain why the additional vowel below either does or does not belong in the phonetic natural class defined by the original set of SNAE vowels.\\

Original set: {[u]}, {[ʊ]}, {[oʊ]}, {[ɔ]}

Addition: {[ɔɪ]}


~\\
INSTRUCTOR NOTES: should recognize that there's more than one vowel sound, which makes it somewhat difficult to categorize; best answers will say that the diphthong is crucially a diphthong and so can't also go in this class


\vfill
Excellent (3) ~~~ Good (2.2) ~~~ Fair (1.7) ~~~ Poor (0)
\newpage

{\large Question 2}\\

Topic: Transcription\\
Source: Week 2 Handout, Part II, Question 11\\

How would this word be transcribed?\\ (Kathleen will then ask a follow-up question about your transcription.)\\

<finger>


~\\
INSTRUCTOR NOTES: [fɪŋɡɹ̩]


\vfill
Excellent (3) ~~~ Good (2.2) ~~~ Fair (1.7) ~~~ Poor (0)
\newpage

\begin{center}
\textbf{{\color{red}{\HUGE END OF EXAM}}}\\

\end{center}
\newpage

\begin{center}
\textbf{{\color{blue}{\HUGE START OF EXAM\\}}}

\textbf{{\color{blue}{\HUGE Student ID: 74115\\}}}

\textbf{{\color{blue}{\HUGE 4:50\\}}}

\end{center}
\newpage

{\large Question 1}\\

Topic: Acoustics\\
Source: Week 9 Handout, Question 7\\

Explain why each numbered, underlined statement is true or false. If it is false, explain one way that you could correct it.\\

$^{10}$\ul{Frequency is inversely related to pitch: high frequencies correspond to low pitches, and low frequencies correspond to high pitches.} Finally, there is the amplitude of the wave. $^{11}$\ul{The amplitude tells you how much pressure the molecules are under at any particular time.} $^{12}$\ul{The auditory correlate of amplitude is intensity}; this is a measure of perceived pressure.\\\\$^{13}$\ul{In speech, air is set in vibrating motion by the lungs, so the lungs} are the source of most speech sounds.


~\\
INSTRUCTOR NOTES: 10 - false (frequency is directly related to pitch: high frequencies correspond to high pitches, and low frequencies correspond to low pitches).\\11 - true.\\12 - false (the auditory correlate of amplitude and intensity is loudness/volume, or, a related acoustic measure to amplitude is intensity).\\13 - false (air is set in vibrating motion by the vocal folds).


\vfill
Excellent (3) ~~~ Good (2.2) ~~~ Fair (1.7) ~~~ Poor (0)
\newpage

{\large Question 2}\\

Topic: Transcription\\
Source: Week 2 Handout, Part II\\

Is this a reasonable transcription of this word? Explain why.\\

<climb>: {[klɑɪm]}


~\\
INSTRUCTOR NOTES: yes


\vfill
Excellent (3) ~~~ Good (2.2) ~~~ Fair (1.7) ~~~ Poor (0)
\newpage

\begin{center}
\textbf{{\color{red}{\HUGE END OF EXAM}}}\\

\end{center}
\newpage

\end{document}

